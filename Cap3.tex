\begin{center}
\chapter{Diseño}
\end{center}
\newpage

En este capítulo se desarrolla los apartados dedicados a la planeación del trabajo a realizar, comenzando con el diseño utilizando elementos de la metodología de Metodología de ingeniería de software multimedia y los elementos de apoyo de la metodología estructurada de Yourdon cuando estos sean necesarios para ejemplificar diseño o acciones específicas del software. 

\section{Selección de estrategia de diseño}
Con respecto a la realidad virtual, una de las cosas básicas a considerar es la forma en que nuestros sentidos sirven como la entrada que nuestro cerebro utiliza para construir una comprensión del mundo que nos rodea. La vista, el oído, el tacto, el olfato y el gusto son el conjunto de estímulo externo más ampliamente aceptado que percibe el cuerpo humano.\\

Estos sentidos y nuestras reacciones a ellos son el resultado de milenios de selección natural y hay varias consecuencias de esto incorporadas en nuestro instinto. Todo esto es un conocimiento relativamente común y parece que no es necesario reiterar aquí, pero lo importante es afirmar que nosotros, como humanos, tenemos ciertos resultados predecibles basados en ciertos conjuntos de entradas. Esencialmente, es instinto, naturaleza humana.\\

Un sitio web bien diseñado utilizará de manera similar el color, la distancia y la tipografía para comunicar claramente un propósito y, a menudo, persuadir algún tipo de acción.\\

Para que todo esto sea efectivo, se deben implementar y descubrir principios de diseño razonables. Existen varios principios para el diseño que pueden traducirse de otros medios. El diseño de impresión, el diseño web, la arquitectura, el diseño de interiores, el teatro, los gráficos en movimiento, etc., tienen elementos que pueden considerarse relevantes y adoptados.[ 36]\\

Al mismo tiempo, el medio de la realidad virtual como propiedades, como la capacidad de intersección del contenido, son únicas.\\

Es por esto que el diseño de un sistema de realidad virtual presenta retos los cuales son difíciles de sobrellevar ya que se tiene que crear una experiencia para el usuario en el sistema mismo lo cual conlleva a la selección de una estrategia de diseño centrada en la UX del usuario.\\

\section{Requisitos para el desarrollo de software para proveer una experiencia de realidad virtual optima.}
\subsection{Los cuatro núcleos del diseño UX para RV}
Se tomaron en cuenta dos consideraciones centrales para el diseño de experiencias de realidad virtual:

