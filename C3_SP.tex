\chapter{Solución Propuesta}

\section{Propuesta de Solución}
Se elaboró un sistema de realidad virtual del sistema digestivo del cuerpo humano que permite interactuar con modelos tridimensionales. La intención es sentar las bases
 para un sistema de apoyo al aprendizaje que sea más práctico \cite{moore1995learning}, sin sustituir a ningún método de estudio tradicional.
\\
\section{Viabilidad} \label{viab}
También se analizó la factibilidad del proyecto en general. Desde el punto de vista técnico se realizó una evaluación de la tecnología actual existente y la posibilidad de 
utilizarla en el desarrollo del sistema. Además de DirectX versión 11 el cuadro \ref{tab:t23} muestra los recursos técnicos necesarios para la ejecución correcta del software:\\
\begin{table}[H]
\centering
\begin{tabular}{|c|c|l|}
\hline
\rowcolor[HTML]{9B9B9B} 
\textbf{Cantidad} &
  \textbf{Recursos} &
  \multicolumn{1}{c|}{\cellcolor[HTML]{9B9B9B}\textbf{Características}} \\ \hline
\cellcolor[HTML]{C0C0C0}1 &
  Computadora Personal de Escritorio &
  \begin{tabular}[c]{@{}l@{}}Tarjeta gráfica discreta, RX480\\ Memoria RAM 16 Gb\\ 5 Puertos USB\\ Procesador de 4 núcleos o mayor\end{tabular} \\ \hline
\cellcolor[HTML]{C0C0C0}1 &
  Sistema de realidad Virtual Oculus Rift &
  \begin{tabular}[c]{@{}l@{}}Visor HMD, controles Touch \\ \\ Sensores Touch.\end{tabular} \\ \hline
\end{tabular}
\caption{ Recursos Técnicos.}
\label{tab:t23}
\end{table}
Económicamente, se determinaron los recursos para desarrollar el sistema así como la comparativa con el uso de cuerpos para su examinación y estudio. Después de un análisis e 
investigación de los costos con la dirección del Área de Morfología en la Escuela Superior de Medicina bajo la asesoría del Dr. Macias Rios se determinó que el costo que se 
tiene para el traslado, mantenimiento, uso e inhumación de los cuerpos es de \$40,000.00 c/u, como se ve en el cuadro \ref{tab:t24}.
\begin{table}[H]
\centering
\begin{tabular}{|
>{\columncolor[HTML]{C0C0C0}}c |l|}
\hline
\multicolumn{2}{|c|}{\cellcolor[HTML]{9B9B9B}\textbf{Costo de uso de cuerpos.}} \\ \hline
Traslado, mantenimiento, uso e inhumación           & \$40,000.00 c/u           \\ \hline
Total                                               & \$40,000.00 c/u           \\ \hline
\end{tabular}
\caption{Costo de cálculo de uso de cuerpos.}
\label{tab:t24}
\end{table}

En el caso del desarrollo e implementación del proyecto se consideró la depreciación, como se observa en el cuadro \ref{tab:t25}.
\begin{table}[H]
\centering
\resizebox{\textwidth}{!}{%
\begin{tabular}{clccccc|c|c|}
\hline
\rowcolor[HTML]{9B9B9B} 
\multicolumn{9}{|c|}{\cellcolor[HTML]{9B9B9B}\textbf{Depreciaciones del Proyecto}} \\ \hline
\rowcolor[HTML]{9B9B9B} 
\multicolumn{4}{|c|}{\cellcolor[HTML]{9B9B9B}\textbf{Equipos de Cómputo}} &
  \multicolumn{5}{c|}{\cellcolor[HTML]{9B9B9B}\textbf{Depreciación}} \\ \hline
\rowcolor[HTML]{9B9B9B} 
\multicolumn{1}{|c|}{\cellcolor[HTML]{9B9B9B}\textbf{Cantidad}} &
  \multicolumn{1}{c|}{\cellcolor[HTML]{9B9B9B}\textbf{Equipos}} &
  \multicolumn{1}{c|}{\cellcolor[HTML]{9B9B9B}\textbf{\begin{tabular}[c]{@{}c@{}}Monto original de \\ Inversión\end{tabular}}} &
  \multicolumn{1}{c|}{\cellcolor[HTML]{9B9B9B}\textbf{\begin{tabular}[c]{@{}c@{}}Valor actual\\  del equipo\end{tabular}}} &
  \multicolumn{1}{c|}{\cellcolor[HTML]{9B9B9B}\textbf{\begin{tabular}[c]{@{}c@{}}Valor a \\ \\ depreciar\end{tabular}}} &
  \multicolumn{1}{c|}{\cellcolor[HTML]{9B9B9B}\textbf{\begin{tabular}[c]{@{}c@{}}\%\\ anual\end{tabular}}} &
  \textbf{\begin{tabular}[c]{@{}c@{}}\%\\ mensual\end{tabular}} &
  \textbf{\begin{tabular}[c]{@{}c@{}}Depresiación\\ mensual\end{tabular}} &
  \textbf{\begin{tabular}[c]{@{}c@{}}Depreciación\\ anual\end{tabular}} \\ \hline
\multicolumn{1}{|c|}{\cellcolor[HTML]{C0C0C0}1} &
  \multicolumn{1}{l|}{\begin{tabular}[c]{@{}l@{}}Computadora de\\ escritorio armada\end{tabular}} &
  \multicolumn{1}{c|}{\$25,054.63} &
  \multicolumn{1}{c|}{\$20,000.00} &
  \multicolumn{1}{c|}{\$5,054.63} &
  \multicolumn{1}{c|}{33.33\%} &
  2.78\% &
  \$ 140.52 &
  \$1,545.72 \\ \hline
\multicolumn{1}{|c|}{\cellcolor[HTML]{C0C0C0}1} &
  \multicolumn{1}{l|}{Laptop HP} &
  \multicolumn{1}{c|}{\$9,999.00} &
  \multicolumn{1}{c|}{\$6,999.00} &
  \multicolumn{1}{c|}{\$3,000.00} &
  \multicolumn{1}{c|}{33.33\%} &
  2.78\% &
  \$ 83.40 &
  \$ 917.40 \\ \hline
\multicolumn{7}{l}{} &
  \textbf{Total:} &
  \$ 2,463.12 \\ \cline{8-9} 
\end{tabular}
}
\caption{Depreciaciones del proyecto.}
\label{tab:t25}
\end{table}

Para ofrecer una experiencia aceptable al momento del uso del equipo de Realidad Virtual y el software se proponen los elementos del cuadro \ref{tab:t26}.
%%
  %Tabla de costos de productos, no se ha logrado realizar correctamente%
%%
\textbf{Pendiente tabla costos}\\
Además, el sistema de Realidad Virtual con sus componentes tiene un costo que se muestra en el cuadro \ref{tab:t27}.
\begin{table}[H]
  \centering
  \begin{tabular}{|c|c|}
  \hline
  \rowcolor[HTML]{9B9B9B} 
  \multicolumn{2}{|c|}{\cellcolor[HTML]{9B9B9B}\textbf{Sistema de Realidad Virtual}} \\ \hline
  \rowcolor[HTML]{9B9B9B} 
  \textbf{Producto}                          & \textbf{Producto}                     \\ \hline
  Visor Oculus Rift                          & Incluido en el paquete                \\ \hline
  Controles Touch Oculus x 2                 & Incluido en el paquete                \\ \hline
  Sensores Oculus x 2                        & Incluido en el paquete                \\ \hline
  Anexos                                     & Incluido en el paquete                \\ \hline
  \textbf{Total:}                            & \$ 8,821.74                           \\ \hline
  \end{tabular}
  \caption{Costos y contenido del sistema de Realidad Virtual.}
  \label{tab:t27}
\end{table}

Se estimaron los sueldos de programador y modelado, como se observa en el cuadro \ref{tab:t28}.\\
\begin{table}[H]
  \centering
  \resizebox{\textwidth}{!}{%
  \begin{tabular}{ccc|c|c|}
  \hline
  \rowcolor[HTML]{9B9B9B} 
  \multicolumn{5}{|c|}{\cellcolor[HTML]{9B9B9B}\textbf{Sueldos}}                                                                                                                                                                                                                                                                    \\ \hline
  \rowcolor[HTML]{9B9B9B} 
  \multicolumn{1}{|c|}{\cellcolor[HTML]{9B9B9B}\textbf{Puesto}} & \multicolumn{1}{c|}{\cellcolor[HTML]{9B9B9B}\textbf{\begin{tabular}[c]{@{}c@{}}Sueldo\\ Mensual\\ individual\end{tabular}}} & \textbf{\begin{tabular}[c]{@{}c@{}}Cantidad \\ \\ de personal\end{tabular}} & \textbf{Sueldos mensuales totales} & \textbf{6 meses} \\ \hline
  \multicolumn{1}{|c|}{Programador}                             & \multicolumn{1}{c|}{\$25,296.00}                                                                                            & 1                                                                           & \$25,296.00                        & \$151,776.00     \\ \hline
  \multicolumn{1}{|c|}{Modelador 3D}                            & \multicolumn{1}{c|}{\$25,296.00}                                                                                            & 1                                                                           & \$25,296.00                        & \$151,776.00     \\ \hline
  \multicolumn{3}{l}{}                                                                                                                                                                                                                                                      & \multicolumn{1}{l|}{Total}         & \$303,522.00     \\ \cline{4-5} 
  \end{tabular}%
  }
  \caption{Cálculo de Sueldos.
  }
  \label{tab:t28}
\end{table}

\begin{table}[H]
  \centering
  \begin{tabular}{c|c|c|}
  \hline
  \rowcolor[HTML]{9B9B9B} 
  \multicolumn{3}{|c|}{\cellcolor[HTML]{9B9B9B}\textbf{Servicios}}                                       \\ \hline
  \rowcolor[HTML]{9B9B9B} 
  \multicolumn{1}{|c|}{\cellcolor[HTML]{9B9B9B}\textbf{Concepto}} & \textbf{Mensual} & \textbf{11 Meses} \\ \hline
  \multicolumn{1}{|c|}{Luz (kw Consumidos por costo Unitario)}    & \$430            & \$4,730           \\ \hline
  \multicolumn{1}{|c|}{Agua (Lt consumidos por costo unitario)}   & \$200            & \$2,200           \\ \hline
  \multicolumn{1}{|c|}{Teléfono e Internet (renta mensual fija)}  & \$ 450           & \$4,850           \\ \hline
  \multicolumn{1}{l|}{}                                           & Total:           & \$11,780          \\ \cline{2-3} 
  \end{tabular}
  
  \caption{Cálculo de Costo por Servicios.}
  \label{tab:t29}
\end{table}

Los servicios estimados se muestran en el cuadro \ref{tab:t29} y en el cuadro \ref{tab:t210} se muestra la suma total y como resultado se obtiene el costo total del proyecto, 
que se estima en: \$326,316.86.\\
\begin{table}[H]
  \centering
  \begin{tabular}{|l|r|}
  \hline
  \rowcolor[HTML]{9B9B9B} 
  \multicolumn{2}{|c|}{\cellcolor[HTML]{9B9B9B}\textbf{Costos del Proyecto}}                                                    \\ \hline
  \rowcolor[HTML]{9B9B9B} 
  \multicolumn{1}{|c|}{\cellcolor[HTML]{9B9B9B}\textbf{Concepto}} & \multicolumn{1}{c|}{\cellcolor[HTML]{9B9B9B}\textbf{Costo}} \\ \hline
  Servicios                                                       & \$ 11,780                                                   \\ \hline
  Sueldos                                                         & \$303,522.00                                                \\ \hline
  Depreciaciones                                                  & \$2,463.12                                                  \\ \hline
  Equipo extra.                                                   & \$ 8,821.74                                                 \\ \hline
  Total                                                           & \$ 326,316.86                                               \\ \hline
  \end{tabular}
  \caption{Costos finales del proyecto}
  \label{tab:t210}
  \end{table}

En resumen, el costo de usar nueve cuerpos sería de \$360,000.00 y el del proyecto de \$326,318.00, con lo cual se puede considerar viable económicamente.\\
