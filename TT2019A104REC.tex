\documentclass[11pt,letter]{report}  

%%%%%%%% PREÁMBULO %%%%%%%%%%%%
\title{Trabajo Terminal}
\usepackage[spanish]{babel} %Indica que escribiermos en español
\usepackage[utf8]{inputenc} %Indica qué codificación se está usando ISO-8859-1(latin1)  o utf8  
\usepackage{graphicx} % Incluir imágenes en LaTeX
\usepackage{color} % Para colorear texto
\usepackage{subfigure} % subfiguras
\usepackage{float} %Podemos usar el especificador [H] en las figuras para que se
% queden donde queramos
\usepackage{capt-of} % Permite usar etiquetas fuera de elementos flotantes
% (etiquetas de figuras)
\usepackage{sidecap} % Para poner el texto de las imágenes al lado
\sidecaptionvpos{figure}{c} % Para que el texto se alinie al centro vertical
\usepackage{caption} % Para poder quitar numeracion de figuras
\usepackage{anysize} % Para personalizar el ancho de  los márgenes
\marginsize{2cm}{2cm}{2cm}{2cm} % Izquierda, derecha, arriba, abajo

\usepackage{appendix}
\renewcommand{\appendixname}{Apéndices}
\renewcommand{\appendixtocname}{Apéndices}
\renewcommand{\appendixpagename}{Apéndices} 

% Para que las referencias sean hipervínculos a las figuras o ecuaciones y
% aparezcan en color
\usepackage[colorlinks=true,plainpages=true,citecolor=blue,linkcolor=blue]{hyperref}
%\usepackage{hyperref} 
% Para agregar encabezado y pie de página
\usepackage{fancyhdr} 
\pagestyle{fancy}
\fancyhf{}
\fancyhead[L]{\footnotesize TT 2019-A104} %encabezado izquierda
\fancyhead[R]{\footnotesize IPN - ESCOM}   % dereecha
\fancyfoot[R]{\footnotesize Trabajo Terminal II}  % Pie derecha
\fancyfoot[C]{\thepage}  % centro
\fancyfoot[L]{\footnotesize Ingeniar\'ia en Sistemas Computacionales}  %izquierda
\renewcommand{\footrulewidth}{0.4pt}


\usepackage{listings} % Para usar código fuente
\definecolor{dkgreen}{rgb}{0,0.6,0} % Definimos colores para usar en el código
\definecolor{gray}{rgb}{0.5,0.5,0.5} 
% configuración para el lenguaje que queramos utilizar
\lstset{language=Matlab,
   keywords={break,case,catch,continue,else,elseif,end,for,function,
      global,if,otherwise,persistent,return,switch,try,while},
   basicstyle=\ttfamily,
   keywordstyle=\color{blue},
   commentstyle=\color{red},
   stringstyle=\color{dkgreen},
   numbers=left,
   numberstyle=\tiny\color{gray},
   stepnumber=1,
   numbersep=10pt,
   backgroundcolor=\color{white},
   tabsize=4,
   showspaces=false,
   showstringspaces=false}

\newcommand{\sen}{\operatorname{\sen}}	% Definimos el comando \sen para el seno
%en español

\title{Plantilla para Trabajo Terminal de ESCOM}

%%%%%%%% TERMINA PREÁMBULO %%%%%%%%%%%%

\begin{document}

%%%%%%%%%%%%%%%%%%%%%%%%%%%%%%%%%% PORTADA %%%%%%%%%%%%%%%%%%%%%%%%%%%%%%%%%%%%%%%%%%%%
\begin{center}
\newcommand{\HRule}{\rule{\linewidth}{0.5mm}}
\begin{minipage}{0.48\textwidth} \begin{flushleft}
\includegraphics[scale = 0.09]{./images/ipn.png}
\end{flushleft}\end{minipage}
\begin{minipage}{0.48\textwidth} \begin{flushright}
\includegraphics[scale = 0.08]{images/escom.png}
\end{flushright}\end{minipage}
\vspace*{1.5cm}
\textsc{\huge Instituto Polit\'ecnico\\ \vspace{5px} Nacional}\\[1.5cm]
\textsc{\LARGE Escuela Superior de C\'omputo}\\[1.5cm]
\begin{minipage}{0.9\textwidth} 
\begin{center}
\textsc{\LARGE Trabajo Terminal II}
\end{center}
\end{minipage}\\[0.5cm]
\vspace*{1cm}
\HRule \\[0.4cm]
{ \huge \bfseries “Sistema de realidad virtual del cuerpo humano para el estudio del sistema digestivo”
}\\[0.4cm]
\HRule \\[1.5cm]
\begin{minipage}{0.46\textwidth}
\begin{flushleft} \large
\emph{Autor:}\\	
Almendarez Perdomo Rodrigo\\
\end{flushleft}
\end{minipage}
\begin{minipage}{0.52\textwidth}		
\vspace{-0.6cm}
\begin{flushright} \large
\emph{Directores:} \\
M. en Ing. Moscoso Malagón Yosafat\\
M. en C. Saucedo Delgado Rafael Norman
\end{flushright}
\end{minipage}
\vspace*{1cm}
\vspace{1cm}	
\emph{Para obtener el t\'itulo de }\\
\vspace{1cm}	
{\textbf{\Large Ing. en Sistemas Computacionales}	}\\
\vspace{1cm}
\begin{center}
{\large \today}
\end{center}											  						
\end{center}

%%%%%%%%%%%%%%%%%%%%%%%%%%%%%%%%%% RESUMEN %%%%%%%%%%%%%%%%%%%%%%%%%%%%%%%%%%%%%%%%%%%%
\begin{abstract}												
\begin{minipage}{0.48\textwidth} \begin{flushleft}
\includegraphics[scale = 0.09]{images/ipn}
\end{flushleft}\end{minipage}
\begin{minipage}{0.48\textwidth} \begin{flushright}
\includegraphics[scale = 0.08]{images/escom}
\end{flushright}\end{minipage}

El presente proyecto consiste en desarrollar un software que está enfocado a la enseñanza, aprendizaje y demostración del cuerpo humano mediante Realidad Virtual (R.V.), el mismo que puede ser usado en la Escuela Superior de Medicina (E.S.M.) y Escuela Nacional de Medicina y Homeopatía (E.N.M. y H.) en las áreas de Morfología y Anatomía como soporte para la enseñanza de las áreas antes mencionadas. 
\\
De esta manera se espera aumentar el número de herramientas que poseen los estudiantes de medicina y reforzar el aprendizaje del alumnado con tecnología actual, escalable, con mayor disponibilidad y facilidad de uso, en contraste con el uso de cadáveres humanos.
\\[4cm]
\textbf{Palabras clave} - Animación por Computadora, Gráficos por Computadora, Realidad Virtual, Estudio Multimedia
\\[2cm]
\textbf{Presenta:}
\\[0.5cm]
Almendarez Perdomo Rodrigo
\\[2cm]
\textbf{Directores}
\\[0.5cm]
M. en Ing. Moscoso Malagón Yosafat
\\
M. en C. Saucedo Delgado Rafael Norman
\\
\end{abstract}							

\newpage	
\section*{Advertencia}
\vfill
\textit{“Este documento contiene información desarrollada por la Escuela Superior de Cómputo del Instituto Politécnico Nacional, a partir de datos y documentos con derecho de propiedad y, por lo tanto, su uso quedará restringido a las aplicaciones que explícitamente se convengan.”
\\
La aplicación no convenida exime a la escuela su responsabilidad técnica y da lugar a las consecuencias legales que para tal efecto se determinen. 
\\
Información adicional sobre este reporte técnico podrá obtenerse en: 
\\
La Subdirección Académica de la Escuela Superior de Cómputo del Instituto Politécnico Nacional, situada en Av. Juan de Dios Bátiz s/n Teléfono: 57296000, extensión 52000.}
\vfill

\newpage
\section*{Agradecimientos}
\vfill
En primer lugar quiero agradecer a mis directores, M. en Ing. Moscoso Malagón Yosafat y el M. en C. Saucedo Delgado Rafael Norman, quienes con sus conocimientos y apoyo me guiaron a través de cada una de las etapas de este proyecto para alcanzar los resultados que buscaba, después de todo el  desarrollo de un Trabajo Terminal es el pináculo de  la formación de un alumno en la carrera de Ingeniería en Sistemas Computacionales, el cual requiere de mucha dedicación y conocimientos para llegar a su desarrollo exitoso.
\\
\newline
Especialmente quiero agradecer al profesor Alfredo Rangel Guzmán por sus enseñanzas sobre la vida y el actuar del mundo actual, y el recibir su apoyo incondicional.
También quiero agradecer a la Escuela Superior de Cómputo por brindarme todos los recursos, herramientas y conocimientos que fueron necesarios para llevar a cabo el proceso de investigación y desarrollo del Trabajo Terminal. No hubiese podido arribar a estos resultados de no haber sido por su incondicional ayuda.
\\
\newline
Por último, quiero agradecer a todos mis compañeros y amigos, principalmente a los que me han acompañado desde el inicio de mi carrera y a mi familia y pareja, por apoyarme aún cuando mis ánimos decaían. En especial, quiero hacer mención de mis padres, que siempre estuvieron ahí para darme palabras de apoyo y un abrazo reconfortante para renovar energías y continuar trabajando para finalizar este camino.
\\
\newline
Muchas gracias a todos.
\\
\paragraph{Rodrigo Almendarez Perdomo}
\vfill

\tableofcontents

\begin{center}
\chapter{Introducci\'on}
\end{center}

\newpage 
\section*{Introducci\'on}

Este documento es el reporte técnico final del trabajo terminal titulado “Sistema de realidad virtual del cuerpo humano para el estudio del sistema digestivo” con número de registro TT: 2019-A104.\\
\newline
En el presente capítulo se habla del problema identificado, por qué se considera como tal y cómo es que se ayudó a resolver el problema planteado mediante la ingeniería en sistemas computacionales. También se menciona que se obtiene al concluir con este trabajo terminal, tales como el prototipo del sistema.\\
\newline
En el capítulo \textbf{II Análisis} se mostrarán todos los diagramas y documentos generados al analizar y generar un diseño del sistema que se estará desarrollando. Aquí se encuentra la arquitectura general del sistema, y los modelos gráficos de apoyo presentados en el Análisis Estructurado Moderno.\\
\newline
En el capítulo \textbf{III Diseño} se describe el trabajo generado en el desarrollo del documento hasta el mes de mayo de 2020 para TT2.\\
\newline
En el capítulo \textbf{IV Verificación y Pruebas} se muestran pruebas hechas sobre las implementaciones del sistema siguiendo un guión para la prueba.
 \newline\\
En el capítulo \textbf{V Conclusión} se muestran los resultados obtenidos y experiencias para mejorar el proceso, así como la vertiente para continuar con el trabajo y las conclusiones del integrante.
\newline\\
Finalmente, se encuentran las referencias de todos los recursos empleados para dar soporte y estructura a este Trabajo Terminal, y en los apéndices se anexan elementos extra que dan información más detallada sobre lo que aquí fue realizado.\\
 

\newpage

\section{Detecci\'oon del problema} 

Durante 2019 se entrevistó al Dr. Rios Macias jefe del área de morfología de la Escuela Superior de Medicina del Instituto Politécnico Nacional y comentó que “los medios que se utilizan para el estudio del cuerpo humano principalmente son medios impresos tradicionales, así como el uso de cuerpos para su disección y análisis posterior”. El uso de cuerpos para su disección tiene un alto costo que incluye el mantenimiento del cuerpo en las instalaciones, el mantenimiento de las instalaciones, y la inhumación de los cuerpos.
\\
\newline
\section{Propuesta de Solución}

Se elaboró un sistema de realidad virtual del sistema digestivo del cuerpo humano que permite interactuar con modelos tridimensionales. La intención es sentar las bases para un sistema de apoyo al aprendizaje que sea más práctico \cite{APA:Ref1}, sin sustituir a ningún método de estudio tradicional.
\\
\newline
\section{Desarrollo de la solución.}

bla bla bla bla bla bla bla bla bla bla bla bla bla bla blabla bla bla bla blabla bla bla bla blabla bla bla bla blabla bla bla bla blabla bla bla bla blabla bla bla bla blabla bla bla bla blabla bla bla bla bla bla bla bla bla blabla bla bla bla bla bla bla bla bla blabla bla bla bla blabla bla bla bla blabla bla bla bla bla bla bla bla bla bla bla bla bla bla bla bla bla bla bla bla bla bla bla bla bla bla bla bla bla bla bla bla bla bla bla



\begin{figure}[H]
	\begin{center}
 		\includegraphics[width = 0.5\textwidth]{images/ipn.png}
 		\captionof{figure}{\label{fig:IPN}Descripción de la imagen} 
	\end{center} 
\end{figure}



\subsection{Generación del concepto.}

bla bla bla bla bla bla bla bla bla bla bla bla bla bla blabla bla bla bla blabla bla bla bla blabla bla bla bla blabla bla bla bla blabla bla bla bla blabla bla bla bla blabla bla bla bla blabla bla bla bla bla bla bla bla bla blabla bla bla bla bla bla bla bla bla blabla bla bla bla blabla bla bla bla blabla bla bla bla bla bla bla bla bla bla bla bla bla bla bla bla bla bla bla bla bla bla bla bla bla bla bla bla bla bla bla bla bla bla bla

bla bla bla bla bla bla bla bla bla bla bla bla bla bla blabla bla bla bla blabla bla bla bla blabla bla bla bla blabla bla bla bla blabla bla bla bla blabla bla bla bla blabla bla bla bla blabla bla bla bla bla bla bla bla bla blabla bla bla bla bla bla bla bla bla blabla bla bla bla blabla bla bla bla blabla bla bla bla bla bla bla bla bla bla bla bla bla bla bla bla bla bla bla bla bla bla bla bla bla bla bla bla bla bla bla bla bla bla bla
bla bla bla bla bla bla bla bla bla bla bla bla bla bla blabla bla bla bla blabla bla bla bla blabla bla bla bla blabla bla bla bla blabla bla bla bla blabla bla bla bla blabla bla bla bla blabla bla bla bla bla bla bla bla bla blabla bla bla bla bla bla bla bla bla blabla bla bla bla blabla bla bla bla blabla bla bla bla bla bla bla bla bla bla bla bla bla bla bla bla bla bla bla bla bla bla bla bla bla bla bla bla bla bla bla bla bla bla bla
bla bla bla bla bla bla bla bla bla bla bla bla bla bla blabla bla bla bla blabla bla bla bla blabla bla bla bla blabla bla bla bla blabla bla bla bla blabla bla bla bla blabla bla bla bla blabla bla bla bla bla bla bla bla bla blabla bla bla bla bla bla bla bla bla blabla bla bla bla blabla bla bla bla blabla bla bla bla bla bla bla bla bla bla bla bla bla bla bla bla bla bla bla bla bla bla bla bla bla bla bla bla bla bla bla bla bla bla bla


bla bla bla bla bla bla bla bla bla bla bla bla bla bla blabla bla bla bla blabla bla bla bla blabla bla bla bla blabla bla bla bla blabla bla bla bla blabla bla bla bla blabla bla bla bla blabla bla bla bla bla bla bla bla bla blabla bla bla bla bla bla bla bla bla blabla bla bla bla blabla bla bla bla blabla bla bla bla bla bla bla bla bla bla bla bla bla bla bla bla bla bla bla bla bla bla bla bla bla bla bla bla bla bla bla bla bla bla bla


\section{Conclusiones.}

\section{Cronograma de trabajo terminal II.}


%%%%%%% Bibliografía %%%%%%%%

\bibliographystyle{apalike} 
\addcontentsline{toc}{section}{Referencias}  
\bibliography{bib/IEEEabrv,bib/IEEEreferencias.bib} 
%%%%%%% Bibliografía %%%%%%%%    

\appendix  
\clearpage % o \cleardoublepage
\addappheadtotoc 
\appendixpage 

\section{Anexos 1.}


bla bla bla bla bla bla bla bla bla bla bla bla bla bla blabla bla bla bla blabla bla bla bla blabla bla bla bla blabla bla bla bla blabla bla bla bla blabla bla bla bla blabla bla bla bla blabla bla bla bla bla bla bla bla bla blabla bla bla bla bla bla bla bla bla blabla bla bla bla blabla bla bla bla blabla bla bla bla bla bla bla bla bla bla bla bla bla bla bla bla bla bla bla bla bla bla bla bla bla bla bla bla bla bla bla bla bla bla bla 

bla bla bla bla bla bla bla bla bla bla bla bla bla bla blabla bla bla bla blabla bla bla bla blabla bla bla bla blabla bla bla bla blabla bla bla bla blabla bla bla bla blabla bla bla bla blabla bla bla bla bla bla bla bla bla blabla bla bla bla bla bla bla bla bla blabla bla bla bla blabla bla bla bla blabla bla bla bla bla bla bla bla bla bla bla bla bla bla bla bla bla bla bla bla bla bla bla bla bla bla bla bla bla bla bla bla bla bla bla 

\section{Anexos 2.}


bla bla bla bla bla bla bla bla bla bla bla bla bla bla blabla bla bla bla blabla bla bla bla blabla bla bla bla blabla bla bla bla blabla bla bla bla blabla bla bla bla blabla bla bla bla blabla bla bla bla bla bla bla bla bla blabla bla bla bla bla bla bla bla bla blabla bla bla bla blabla bla bla bla blabla bla bla bla bla bla bla bla bla bla bla bla bla bla bla bla bla bla bla bla bla bla bla bla bla bla bla bla bla bla bla bla bla bla bla
bla bla bla bla bla bla bla bla bla bla bla bla bla bla blabla bla bla bla blabla bla bla bla blabla bla bla bla blabla bla bla bla blabla bla bla bla blabla bla bla bla blabla bla bla bla blabla bla bla bla bla bla bla bla bla blabla bla bla bla bla bla bla bla bla blabla bla bla bla blabla bla bla bla blabla bla bla bla bla bla bla bla bla bla bla bla bla bla bla bla bla bla bla bla bla bla bla bla bla bla bla bla bla bla bla bla bla bla bla
bla bla bla bla bla bla bla bla bla bla bla bla bla bla blabla bla bla bla blabla bla bla bla blabla bla bla bla blabla bla bla bla blabla bla bla bla blabla bla bla bla blabla bla bla bla blabla bla bla bla bla bla bla bla bla blabla bla bla bla bla bla bla bla bla blabla bla bla bla blabla bla bla bla blabla bla bla bla bla bla bla bla bla bla bla bla bla bla bla bla bla bla bla bla bla bla bla bla bla bla bla bla bla bla bla bla bla bla bla
bla bla bla bla bla bla bla bla bla bla bla bla bla bla blabla bla bla bla blabla bla bla bla blabla bla bla bla blabla bla bla bla blabla bla bla bla blabla bla bla bla blabla bla bla bla blabla bla bla bla bla bla bla bla bla blabla bla bla bla bla bla bla bla bla blabla bla bla bla blabla bla bla bla blabla bla bla bla bla bla bla bla bla bla bla bla bla bla bla bla bla bla bla bla bla bla bla bla bla bla bla bla bla bla bla bla bla bla bla


bla bla bla bla bla bla bla bla bla bla bla bla bla bla blabla bla bla bla blabla bla bla bla blabla bla bla bla blabla bla bla bla blabla bla bla bla blabla bla bla bla blabla bla bla bla blabla bla bla bla bla bla bla bla bla blabla bla bla bla bla bla bla bla bla blabla bla bla bla blabla bla bla bla blabla bla bla bla bla bla bla bla bla bla bla bla bla bla bla bla bla bla bla bla bla bla bla bla bla bla bla bla bla bla bla bla bla bla bla
bla bla bla bla bla bla bla bla bla bla bla bla bla bla blabla bla bla bla blabla bla bla bla blabla bla bla bla blabla bla bla bla blabla bla bla bla blabla bla bla bla blabla bla bla bla blabla bla bla bla bla bla bla bla bla blabla bla bla bla bla bla bla bla bla blabla bla bla bla blabla bla bla bla blabla bla bla bla bla bla bla bla bla bla bla bla bla bla bla bla bla bla bla bla bla bla bla bla bla bla bla bla bla bla bla bla bla bla bla

\end{document}