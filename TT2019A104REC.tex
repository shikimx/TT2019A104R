\documentclass[11pt]{report}  

%%%%%%%% PREÁMBULO %%%%%%%%%%%%
\title{Trabajo Terminal}
\usepackage[spanish]{babel} %Indica que escribiermos en español
\usepackage[utf8]{inputenc} %Indica qué codificación se está usando ISO-8859-1(latin1)  o utf8  
\usepackage{graphicx} % Incluir imágenes en LaTeX
\usepackage{color} % Para colorear texto
\usepackage{subfigure} % subfiguras
\usepackage{float} %Podemos usar el especificador [H] en las figuras para que se
% queden donde queramos
\usepackage{capt-of} % Permite usar etiquetas fuera de elementos flotantes
% (etiquetas de figuras)
\usepackage{sidecap} % Para poner el texto de las imágenes al lado
\sidecaptionvpos{figure}{c} % Para que el texto se alinie al centro vertical
\usepackage{caption} % Para poder quitar numeracion de figuras
\captionsetup{labelfont={color=gray},font={color=gray}} %Cambia el color de los pies de foto/tabla
\usepackage{anysize} % Para personalizar el ancho de  los márgenes
\marginsize{2cm}{2cm}{2cm}{2cm} % Izquierda, derecha, arriba, abajo

%Paquetes para tablas%
\usepackage{multirow}
\usepackage{graphicx}
\usepackage[table,xcdraw]{xcolor}
\usepackage{longtable}
%https://www.tablesgenerator.com/

\usepackage{appendix}
\renewcommand{\appendixname}{Apéndices}
\renewcommand{\appendixtocname}{Apéndices}
\renewcommand{\appendixpagename}{Apéndices} 

% Para que las referencias sean hipervínculos a las figuras o ecuaciones y
% aparezcan en color
\usepackage[colorlinks=true,plainpages=true,citecolor=blue,linkcolor=blue]{hyperref}
%\usepackage{hyperref} 
% Para agregar encabezado y pie de página
\usepackage{fancyhdr} 
\pagestyle{fancy}
\fancyhf{}
\fancyhead[L]{\footnotesize TT 2019-A104} %encabezado izquierda
\fancyhead[R]{\footnotesize IPN - ESCOM}   % dereecha
\fancyfoot[R]{\footnotesize Trabajo Terminal II}  % Pie derecha
\fancyfoot[C]{\thepage}  % centro
\fancyfoot[L]{\footnotesize Ingeniar\'ia en Sistemas Computacionales}  %izquierda
\renewcommand{\footrulewidth}{0.4pt}


\usepackage{listings} % Para usar código fuente
\definecolor{dkgreen}{rgb}{0,0.6,0} % Definimos colores para usar en el código
\definecolor{gray}{rgb}{0.5,0.5,0.5} 
% configuración para el lenguaje que queramos utilizar



\title{Plantilla para Trabajo Terminal de ESCOM}

%%%%%%%% TERMINA PREÁMBULO %%%%%%%%%%%%

\begin{document}

%%%%%%%%%%%%%%%%%%%%%%%%%%%%%%%%%% PORTADA %%%%%%%%%%%%%%%%%%%%%%%%%%%%%%%%%%%%%%%%%%%%
\begin{center}
\newcommand{\HRule}{\rule{\linewidth}{0.5mm}}
\begin{minipage}{0.48\textwidth} \begin{flushleft}
\includegraphics[scale = 0.09]{./images/ipn.png}
\end{flushleft}\end{minipage}
\begin{minipage}{0.48\textwidth} \begin{flushright}
\includegraphics[scale = 0.08]{images/escom.png}
\end{flushright}\end{minipage}
\vspace*{1.5cm}
\textsc{\huge Instituto Polit\'ecnico\\ \vspace{5px} Nacional}\\[1.5cm]
\textsc{\LARGE Escuela Superior de C\'omputo}\\[1.5cm]
\begin{minipage}{0.9\textwidth} 
\begin{center}
\textsc{\LARGE Trabajo Terminal II}
\end{center}
\end{minipage}\\[0.5cm]
\vspace*{1cm}
\HRule \\[0.4cm]
{ \huge \bfseries “Sistema de realidad virtual del cuerpo humano para el estudio del sistema digestivo”
}\\[0.4cm]
\HRule \\[1.5cm]
\begin{minipage}{0.46\textwidth}
\begin{flushleft} \large
\emph{Autor:}\\	
Almendarez Perdomo Rodrigo\\
\end{flushleft}
\end{minipage}
\begin{minipage}{0.52\textwidth}		
\vspace{-0.6cm}
\begin{flushright} \large
\emph{Directores:} \\
M. en Ing. Moscoso Malagón Yosafat\\
M. en C. Saucedo Delgado Rafael Norman
\end{flushright}
\end{minipage}
\vspace*{1cm}
\vspace{1cm}	
\emph{Para obtener el t\'itulo de }\\
\vspace{1cm}	
{\textbf{\Large Ing. en Sistemas Computacionales}	}\\
\vspace{1cm}
\begin{center}
{\large \today}
\end{center}											  						
\end{center}

%%%%%%%%%%%%%%%%%%%%%%%%%%%%%%%%%% RESUMEN %%%%%%%%%%%%%%%%%%%%%%%%%%%%%%%%%%%%%%%%%%%%
\begin{abstract}												
\begin{minipage}{0.48\textwidth} \begin{flushleft}
\includegraphics[scale = 0.09]{images/ipn}
\end{flushleft}\end{minipage}
\begin{minipage}{0.48\textwidth} \begin{flushright}
\includegraphics[scale = 0.08]{images/escom}
\end{flushright}\end{minipage}

El presente proyecto consiste en desarrollar un software que está enfocado a la enseñanza, aprendizaje y demostración del cuerpo humano mediante Realidad Virtual (R.V.), el mismo que puede ser usado en la Escuela Superior de Medicina (E.S.M.) y Escuela Nacional de Medicina y Homeopatía (E.N.M. y H.) en las áreas de Morfología y Anatomía como soporte para la enseñanza de las áreas antes mencionadas. 
\\
De esta manera se espera aumentar el número de herramientas que poseen los estudiantes de medicina y reforzar el aprendizaje del alumnado con tecnología actual, escalable, con mayor disponibilidad y facilidad de uso, en contraste con el uso de cadáveres humanos.
\\[4cm]
\textbf{Palabras clave} - Animación por Computadora, Gráficos por Computadora, Realidad Virtual, Estudio Multimedia
\\[2cm]
\textbf{Presenta:}
\\[0.5cm]
Almendarez Perdomo Rodrigo
\\[2cm]
\textbf{Directores}
\\[0.5cm]
M. en Ing. Moscoso Malagón Yosafat
\\
M. en C. Saucedo Delgado Rafael Norman
\\
\end{abstract}							

\newpage	
\section*{Advertencia}
\vfill
\textit{“Este documento contiene información desarrollada por la Escuela Superior de Cómputo del Instituto Politécnico Nacional, a partir de datos y documentos con derecho de propiedad y, por lo tanto, su uso quedará restringido a las aplicaciones que explícitamente se convengan.”
\\
La aplicación no convenida exime a la escuela su responsabilidad técnica y da lugar a las consecuencias legales que para tal efecto se determinen. 
\\
Información adicional sobre este reporte técnico podrá obtenerse en: 
\\
La Subdirección Académica de la Escuela Superior de Cómputo del Instituto Politécnico Nacional, situada en Av. Juan de Dios Bátiz s/n Teléfono: 57296000, extensión 52000.}
\vfill

\newpage
\section*{Agradecimientos}
\vfill
En primer lugar quiero agradecer a mis directores, M. en Ing. Moscoso Malagón Yosafat y el M. en C. Saucedo Delgado Rafael Norman, quienes con sus conocimientos y apoyo me guiaron a través de cada una de las etapas de este proyecto para alcanzar los resultados que buscaba, después de todo el  desarrollo de un Trabajo Terminal es el pináculo de  la formación de un alumno en la carrera de Ingeniería en Sistemas Computacionales, el cual requiere de mucha dedicación y conocimientos para llegar a su desarrollo exitoso.
\\
\newline
Especialmente quiero agradecer al profesor Alfredo Rangel Guzmán por sus enseñanzas sobre la vida y el actuar del mundo actual, y el recibir su apoyo incondicional.
También quiero agradecer a la Escuela Superior de Cómputo por brindarme todos los recursos, herramientas y conocimientos que fueron necesarios para llevar a cabo el proceso de investigación y desarrollo del Trabajo Terminal. No hubiese podido arribar a estos resultados de no haber sido por su incondicional ayuda.
\\
\newline
Por último, quiero agradecer a todos mis compañeros y amigos, principalmente a los que me han acompañado desde el inicio de mi carrera y a mi familia y pareja, por apoyarme aún cuando mis ánimos decaían. En especial, quiero hacer mención de mis padres, que siempre estuvieron ahí para darme palabras de apoyo y un abrazo reconfortante para renovar energías y continuar trabajando para finalizar este camino.
\\
\newline
Muchas gracias a todos.
\\
\paragraph{Rodrigo Almendarez Perdomo}
\vfill

\tableofcontents

\begin{center}
\chapter{Introducci\'on}
\end{center}

\newpage 
\section*{Introducci\'on}
\vfill
Este documento es el reporte técnico final del trabajo terminal titulado “Sistema de realidad virtual del cuerpo humano para el estudio del sistema digestivo” con número de registro TT: 2019-A104.\\
\newline
En el presente capítulo se habla del problema identificado, por qué se considera como tal y cómo es que se ayudó a resolver el problema planteado mediante la ingeniería en sistemas computacionales. También se menciona que se obtiene al concluir con este trabajo terminal, tales como el prototipo del sistema.\\
\newline
En el capítulo \textbf{II Análisis} se mostrarán todos los diagramas y documentos generados al analizar y generar un diseño del sistema que se estará desarrollando. Aquí se encuentra la arquitectura general del sistema, y los modelos gráficos de apoyo presentados en el Análisis Estructurado Moderno.\\
\newline
En el capítulo \textbf{III Diseño} se describe el trabajo generado en el desarrollo del documento hasta el mes de mayo de 2020 para TT2.\\
\newline
En el capítulo \textbf{IV Verificación y Pruebas} se muestran pruebas hechas sobre las implementaciones del sistema siguiendo un guión para la prueba.
 \newline\\
En el capítulo \textbf{V Conclusión} se muestran los resultados obtenidos y experiencias para mejorar el proceso, así como la vertiente para continuar con el trabajo y las conclusiones del integrante.
\newline\\
Finalmente, se encuentran las referencias de todos los recursos empleados para dar soporte y estructura a este Trabajo Terminal, y en los apéndices se anexan elementos extra que dan información más detallada sobre lo que aquí fue realizado.\\
\vfill 

\newpage

\section{Detecci\'oon del problema} 

Durante 2019 se entrevistó al Dr. Rios Macias jefe del área de morfología de la Escuela Superior de Medicina del Instituto Politécnico Nacional y comentó que “los medios que se utilizan para el estudio del cuerpo humano principalmente son medios impresos tradicionales, así como el uso de cuerpos para su disección y análisis posterior”. El uso de cuerpos para su disección tiene un alto costo que incluye el mantenimiento del cuerpo en las instalaciones, el mantenimiento de las instalaciones, y la inhumación de los cuerpos.
\\
\newline
\section{Propuesta de Solución}

Se elaboró un sistema de realidad virtual del sistema digestivo del cuerpo humano que permite interactuar con modelos tridimensionales. La intención es sentar las bases para un sistema de apoyo al aprendizaje que sea más práctico \cite{moore1995learning}, sin sustituir a ningún método de estudio tradicional.
\\
\newline
\section{Justificación}
\label{just}

El sistema tiene los siguientes beneficios para docentes y alumnos: 
\begin{itemize}
\item Fuentes confiables. Se usaron textos médicos y sitios web especializados
\item Uso de realidad virtual\cite{norton1994integrating}.
\end{itemize}

El sistema es un ejemplo de cómo se pueden actualizar las herramientas educativas, como las que se pueden encontrar en Statista \cite{web1}, y que representan un mercado de aproximadamente 18.8 mil millones de dólares para el año 2020. Además, Nielsen\cite{web2} muestra los datos sobre la expectativa de adopción de la realidad virtual y aumentada en diferentes continentes, como se muestra el Figura 1 y 2.

\begin{figure}[H]
	\begin{center}
 		\includegraphics[width = 0.7\textwidth]{source/images/image2.png}
 		\captionof{figure}{\label{fig:graph1}Disposición de los consumidores a nivel mundial para usar la realidad virtual y aumentada si esta se encuentra disponible en los próximos 2 años (2020 -2021)}
	\end{center} 
\end{figure}

\begin{figure}[H]
	\begin{center}
 		\includegraphics[width = 0.7\textwidth]{source/images/image10.png}
 		\captionof{figure}{\label{fig:graph2}Disposición de los consumidores a nivel mundial para no  usar la realidad virtual y aumentada si esta se encuentra disponible en los próximos 2 años(2020 -2021)}
	\end{center} 
\end{figure}

A medida que crece el consumo de software, multimedia y videojuegos se aumentará la disponibilidad de sistemas como el que se propone en este trabajo terminal.
\\

\section{Objetivos del Trabajo}
Objetivo es analizar, diseñar, desarrollar y probar un sistema de demostración que utiliza la tecnología de Realidad Virtual, para ofrecer una experiencia orientada al estudio de la anatomía y morfología del cuerpo humano, específicamente del sistema digestivo.
\newline

\subsection{Objetivos Específicos}
Para lograr el objetivo se identificaron los siguientes objetivos específicos:
\newline
\begin{itemize}
\item Investigar en fuentes confiables sobre el sistema digestivo.
\item Analizar, diseñar, desarrollar y probar un entorno de Realidad Virtual para la interacción.
\end{itemize}

\section{Población Objetivo}
De acuerdo a un estudio de Ericsson ConsumerLabe\cite{web3}, en el año 2020 un tercio de los consumidores serán usuarios de Realidad Virtual. Ya desde 2017 se había explorado el nivel de interés de los consumidores en Realidad Virtual\cite{web4} con su potencial de reunir gente de todo el mundo y crear un experiencia más profunda, personalizada y enriquecida.\\
\newline
De todos los interesados en la Realidad Virtual se toma a los estudiantes de educación superior del área de medicina, particularmente a 180\cite{ofi1}  estudiantes de la Escuela Superior de Medicina del Instituto Politécnico Nacional de entre 18 y 24 años de edad. Además se espera que los usuarios objetivo tengan las siguientes características\cite{web5} específicamente:\\
\newline
\begin{itemize}
\item Disposición para el aprendizaje con nuevas tecnologías de la información.
\item Conocimientos sólidos en las áreas de biología, física, química; y en forma idónea conocimientos básicos de las etimologías grecolatinas e idioma inglés, que le facilitarán la comprensión y dominio de los conceptos utilizados en las asignaturas básicas y clínicas.
\item No ser propenso a sufrir cinetosis.
\end{itemize}

\section{Productos logrados}
Se logró crear un software (demostración), de una experiencia demostrativa en realidad virtual. El sistema muestra características y elementos del sistema digestivo y permite la participación de un usuario utilizando una misma computadora de forma local. Se usan los controles Oculus Touch ® así como el visor de Realidad Virtual Oculus Rift ®. También se redactó el presente reporte técnico.

\section{Definiciones}
Estas son las definiciones más importantes para el presente trabajo terminal

\subsection{Realidad Virtual}
Algunos autores definen así la Realidad Virtual.\\
\newline
\textit{“La realidad virtual (RV) es una simulación tridimensional generada o asistida comúnmente por computadora de algún aspecto del mundo real o ficticio, en el cual el usuario tiene la sensación de pertenecer a ese ambiente sintético o interactuar con él”}\cite{web6}\\ 
\textbf{Corrado Padila Érica}\\
\newline
“Realidad Virtual: gráficos 3D en entornos inmersivos que usan I/O
artefactos como guantes, cascos, etc. en busca de mayores grados de iteración
con el ambiente virtual”\cite{web7}\\ 
Lozano Miguel, Calderón\\
\newline
"Realidad Virtual es una forma en que los seres humanos puedan
visualizar, manipular e interactuar con las computadoras y datos extremadamente
complejos".\cite{web8}\\
Isdale, Jerry\\
\newline
“Un sistema interactivo capaz de crear una simulación que implique a varios de los sentidos del ser humano, generados por una computadora, explorable, visualizable y manipulable en tiempo real; este bajo la forma de imágenes y sonidos, estos, dando la sensación de presencia en el entorno generado”\cite{web9}\\
Levis, Diego\\
\newline
Esta última ha sido la definición que se ha tomado para el desarrollo del proyecto del trabajo terminal, asimismo se puede concluir que todos los autores coinciden en que la realidad virtual es un mundo simulado en el que el usuario puede interactuar en tiempo real por medio
de dispositivos o computadoras que logran un efecto artificial e inmersivo en el que se pueden manipular objetos.

\subsection{Producto Multimedia}
Los productos multimedia se pueden clasificar en dos categorías: productos interactivos y no interactivos. Los productos no interactivos también se pueden clasificar en productos estáticos como carteles, logotipos, folletos, modelos estáticos 3D, etc., y productos basados en el tiempo. \cite{engels2002object,sauer2001uml}.\\ 
\newline
Los productos multimedia interactivos son aplicaciones de software que contienen productos multimedia\cite{miranda2017diseno} (es decir, aplicaciones basadas en eventos como juegos, aplicaciones web basadas en multimedia y materiales de aprendizaje multimedia basados en interactividad).  La figura \ref{fig:diag1} a continuación ilustra los tipos de productos multimedia.\\
\begin{figure}[H]
	\begin{center}
 		\includegraphics[width = 0.5\textwidth]{source/images/image52.png}
 		\captionof{figure}{\label{fig:diag1}Tipos de productos multimedia} 
	\end{center} 
\end{figure}

Un sistema multimedia se puede analizar desde tres puntos:
\begin{itemize}
\item La vista externa. La forma en que el usuario interactúa con el producto.
\item Flujo de acciones. El orden en que se muestran los marcos de los modelos a los usuarios. \cite{aleem2016game,cartwright1996pre}.
\item  Los roles de los usuarios. La interacción con productos multimedia.
\end{itemize} 
El cuadro \ref{tab:tab1}1} resume los tipos de productos multimedia y sus características.

\begin{longtable}[c]{
>{\columncolor[HTML]{EFEFEF}}c ccc}
\cellcolor[HTML]{C0C0C0} &
  \multicolumn{3}{c}{\cellcolor[HTML]{C0C0C0}\textbf{Características Multimedia}} \\
\multirow{-2}{*}{\cellcolor[HTML]{C0C0C0}\textbf{\begin{tabular}[c]{@{}c@{}}Tipos de productos\\  multimedia\end{tabular}}} &
  \cellcolor[HTML]{C0C0C0}\textbf{\begin{tabular}[c]{@{}c@{}}Vista \\ \\ Externa\end{tabular}} &
  \cellcolor[HTML]{C0C0C0}\textbf{\begin{tabular}[c]{@{}c@{}}Flujo \\ \\ de acciones\end{tabular}} &
  \cellcolor[HTML]{C0C0C0}\textbf{\begin{tabular}[c]{@{}c@{}}Roles de \\ \\ los Usuarios\end{tabular}} \\
\endfirsthead
%
\endhead
%
\hline
\endfoot
%
\endlastfoot
%
\textit{Productos estáticos} &
  Un cuadro &
  Sin acciones &
  Pasivo (ver/leer) \\
\textit{\begin{tabular}[c]{@{}c@{}}Productos basados\\  en tiempo\end{tabular}} &
  Múltiples cuadros &
  Secuencia &
  Pasivo(mirar) \\
\textit{Productos interactivos} &
  Impulsado por eventos &
  \begin{tabular}[c]{@{}c@{}}Secuencia, Selectiva,\\  Interactiva e\\  Impulsada por eventos\end{tabular} &
  Activo(Realiza eventos) \\ \hline
\caption{Características de los productos multimedia.}
\label{tab:tab1}\\
\end{longtable}
Para este trabajo, el sistema que se propone es una producto interactivo.

\section{Estado del Arte}
A continuación se muestran algunos de trabajos académicos desarrollados en México y fueron comparados con el trabajo planeado. Como comparativa y de forma ilustrativa del sector académico.
\\
\newline
\begin{enumerate}
\item TT No. 2014-A058 “Sistema para la orientación de los efectos sobre la espalda humana en pacientes con sobrepeso” [ 18 ]
\item TT No. 2012-B055 “Laboratorio Virtual del cuerpo humano 3D con asistente de ayuda en línea para el nivel superior bajo el paradigma de Educación Basada en Web con tecnologías de Web Semántica” [ 19 ]
\item TT No. 2014-B035 “Simulación en Tercera Dimensión del Sistema Circulatorio de los Cánidos para el uso Educativo” [ 20 ]
\item TT No. 2014-B039 “Simulación de una Línea del Metro con Realidad Virtual” [ 21 ]
\item Tesis que para optar por el grado de Maestro en Ciencia e Ingeniería de la Computación, Sistema de seguimiento de movimiento de las extremidades superiores basado en sensores inerciales para rehabilitación en realidad virtual. [ 22 ]
\item Adecuación educativa de la realidad virtual como herramienta didáctica para el proceso enseñanza-aprendizaje / tesis que para obtener el título de Licenciado en Pedagogía, presenta Maria de la O García Noriega; asesor Lucina Moreno Valle Suárez.  [ 23 ]
\end{enumerate}
Así mismo se ha encontrado software propietario desarrollado por empresas privadas los cuales son los siguientes.\\
\begin{enumerate}
\item The Body VR: Anatomy Viewer es la única herramienta de visualización de Realidad Virtual disponible en el mercado que se basa en datos médicos específicos del paciente (por ejemplo, MRI, CT, PET) y cumple con los estándares DICOM. Proporciona simulaciones de R.V. anatómicas en tiempo real para visualizar diagnósticos médicos, ilustrar el impacto de los procedimientos y tratamientos, y crear una toma de decisiones más educada.

Figura 3 - Software “The Body VR” en uso.

\item Anatomyou VR: Estructuras anatómicas fotorrealistas, modeladas en colaboración con RenderArea, validadas por expertos clínicos y certificadas por personal capacitado en  Tecnologías Médicas de la Universidad de Las Palmas de Gran Canaria.

Figura 4 -  Pancarta promocional de “Anatomyou VR” 

\item Biodigital Anatomy: El cuerpo tridimensional más completo, científicamente preciso e interactivo jamás ensamblado. Anatomía masculina y femenina, en los detalles básicos (gratuitos) y profesionales. Cada sistema está completamente segmentado, etiquetado y direccionable para una fácil configuración que satisfaga cualquier necesidad educativa.


Figura 5 - Interfaz del software de “Biodigital Anatomy”

\item 3D Organon VR Anatom: 3D Organon es un completo atlas anatómico que presenta los 15 sistemas del cuerpo humano. Incluye más de 4,000 estructuras y órganos anatómicos realistas y más de 160 correlaciones clínicas encontradas por sistema del cuerpo.

\end{enumerate}

%\begin{figure}[H]
%	\begin{center}
% 		\includegraphics[scale=•]{•} = 0.5\textwidth]{images/ipn.png}
% 		\captionof{figure}{\label{fig:IPN}Descripción de la %imagen} 
%	\end{center} 
%\end{figure}





%%%%%%% Bibliografía %%%%%%%%
\bibliography{Biblio.bib} 
\bibliographystyle{ieeetr} 
\addcontentsline{toc}{section}{Referencias}  

%%%%%%% Bibliografía %%%%%%%%    

\appendix  
\clearpage % o \cleardoublepage
\addappheadtotoc 
\appendixpage 

\section{Anexos}


bla bla bla bla bla bla bla bla bla bla bla bla bla bla blabla bla bla bla blabla bla bla bla blabla bla bla bla blabla bla bla bla blabla bla bla bla blabla bla bla bla blabla bla bla bla blabla bla bla bla bla bla bla bla bla blabla bla bla bla bla bla bla bla bla blabla bla bla bla blabla bla bla bla blabla bla bla bla bla bla bla bla bla bla bla bla bla bla bla bla bla bla bla bla bla bla bla bla bla bla bla bla bla bla bla bla bla bla bla 

bla bla bla bla bla bla bla bla bla bla bla bla bla bla blabla bla bla bla blabla bla bla bla blabla bla bla bla blabla bla bla bla blabla bla bla bla blabla bla bla bla blabla bla bla bla blabla bla bla bla bla bla bla bla bla blabla bla bla bla bla bla bla bla bla blabla bla bla bla blabla bla bla bla blabla bla bla bla bla bla bla bla bla bla bla bla bla bla bla bla bla bla bla bla bla bla bla bla bla bla bla bla bla bla bla bla bla bla bla 

\section{Anexos}


bla bla bla bla bla bla bla bla bla bla bla bla bla bla blabla bla bla bla blabla bla bla bla blabla bla bla bla blabla bla bla bla blabla bla bla bla blabla bla bla bla blabla bla bla bla blabla bla bla bla bla bla bla bla bla blabla bla bla bla bla bla bla bla bla blabla bla bla bla blabla bla bla bla blabla bla bla bla bla bla bla bla bla bla bla bla bla bla bla bla bla bla bla bla bla bla bla bla bla bla bla bla bla bla bla bla bla bla bla
bla bla bla bla bla bla bla bla bla bla bla bla bla bla blabla bla bla bla blabla bla bla bla blabla bla bla bla blabla bla bla bla blabla bla bla bla blabla bla bla bla blabla bla bla bla blabla bla bla bla bla bla bla bla bla blabla bla bla bla bla bla bla bla bla blabla bla bla bla blabla bla bla bla blabla bla bla bla bla bla bla bla bla bla bla bla bla bla bla bla bla bla bla bla bla bla bla bla bla bla bla bla bla bla bla bla bla bla bla
bla bla bla bla bla bla bla bla bla bla bla bla bla bla blabla bla bla bla blabla bla bla bla blabla bla bla bla blabla bla bla bla blabla bla bla bla blabla bla bla bla blabla bla bla bla blabla bla bla bla bla bla bla bla bla blabla bla bla bla bla bla bla bla bla blabla bla bla bla blabla bla bla bla blabla bla bla bla bla bla bla bla bla bla bla bla bla bla bla bla bla bla bla bla bla bla bla bla bla bla bla bla bla bla bla bla bla bla bla
bla bla bla bla bla bla bla bla bla bla bla bla bla bla blabla bla bla bla blabla bla bla bla blabla bla bla bla blabla bla bla bla blabla bla bla bla blabla bla bla bla blabla bla bla bla blabla bla bla bla bla bla bla bla bla blabla bla bla bla bla bla bla bla bla blabla bla bla bla blabla bla bla bla blabla bla bla bla bla bla bla bla bla bla bla bla bla bla bla bla bla bla bla bla bla bla bla bla bla bla bla bla bla bla bla bla bla bla bla


bla bla bla bla bla bla bla bla bla bla bla bla bla bla blabla bla bla bla blabla bla bla bla blabla bla bla bla blabla bla bla bla blabla bla bla bla blabla bla bla bla blabla bla bla bla blabla bla bla bla bla bla bla bla bla blabla bla bla bla bla bla bla bla bla blabla bla bla bla blabla bla bla bla blabla bla bla bla bla bla bla bla bla bla bla bla bla bla bla bla bla bla bla bla bla bla bla bla bla bla bla bla bla bla bla bla bla bla bla
bla bla bla bla bla bla bla bla bla bla bla bla bla bla blabla bla bla bla blabla bla bla bla blabla bla bla bla blabla bla bla bla blabla bla bla bla blabla bla bla bla blabla bla bla bla blabla bla bla bla bla bla bla bla bla blabla bla bla bla bla bla bla bla bla blabla bla bla bla blabla bla bla bla blabla bla bla bla bla bla bla bla bla bla bla bla bla bla bla bla bla bla bla bla bla bla bla bla bla bla bla bla bla bla bla bla bla bla bla

\end{document}