\begin{center}
\chapter{An\'alisis}
\end{center}
\newpage

\section{Sistema operativo y platadorma de desarrollo}
EL cuadro \ref{tab:t21} es la comparativa de las plataformas que se analizaron para la elaboración del trabajo terminal.

\begin{table}[H]
\centering
\resizebox{\textwidth}{!}{%
\begin{tabular}{
>{\columncolor[HTML]{C0C0C0}}l lll}
\hline
\multicolumn{4}{c}{\cellcolor[HTML]{9B9B9B}{\color[HTML]{333333} \textbf{Plataforma Windows}}} \\
\multicolumn{1}{c}{\cellcolor[HTML]{9B9B9B}{\color[HTML]{333333} \textbf{Descripción}}} &
  \multicolumn{1}{c}{\cellcolor[HTML]{9B9B9B}{\color[HTML]{333333} \textbf{Software}}} &
  \multicolumn{1}{c}{\cellcolor[HTML]{9B9B9B}{\color[HTML]{333333} \textbf{Costo}}} &
  \multicolumn{1}{c}{\cellcolor[HTML]{9B9B9B}{\color[HTML]{333333} \textbf{Operatividad}}} \\
Sistema Operativo &
  Windows 10 Pro &
  MXN\$5,199.00 &
  \begin{tabular}[c]{@{}l@{}}Ideal para pequeñas empresas o\\  usuarios que necesiten una\\  funcionalidad mejorada.\end{tabular} \\
Sistema Operativo &
  Windows 10 Home &
  MXN\$3,599.00 &
  Ideal para uso personal o doméstico. \\
Sistema Operativo &
  Windows 10 Pro para Workstations &
  MXN\$7,899.00 &
  \begin{tabular}[c]{@{}l@{}}Ideal para los usuarios avanzados y\\  pequeñas empresas que buscan \\ funcionalidad mejorada con la \\ capacidad de calcular cargas de\\  trabajo intensivas.\end{tabular} \\
Sistema Operativo &
  Windows Server 2019 &
  MXN\$10,814.69 &
  \begin{tabular}[c]{@{}l@{}}Sistema operativo que une los entornos\\  on-premises con Azure y agrega capas\\  adicionales de seguridad a la vez que\\  te ayuda a modernizar tus aplicaciones\\  e infraestructura.\end{tabular}
\end{tabular}%
}
\caption{Sistemas operativos de plataforma Windows}
\label{tab:t21}
\end{table}

El cuadro \ref{tab:t22} muestra la comparativa de las plataformas de desarrollo que se analizaron para la elaboración del trabajo terminal.

\begin{table}[H]
\resizebox{\textwidth}{!}{%
\begin{tabular}{|c|c|c|l|}
\hline
\rowcolor[HTML]{9B9B9B} 
\multicolumn{4}{|c|}{\cellcolor[HTML]{9B9B9B}\textbf{Plataforma de desarrollo}}                                                \\ \hline
\rowcolor[HTML]{9B9B9B} 
\textbf{Descripción} & \textbf{Software} & \textbf{Costo} & \multicolumn{1}{c|}{\cellcolor[HTML]{9B9B9B}\textbf{Operatividad}} \\ \hline
\cellcolor[HTML]{C0C0C0}Motor de Desarrollo &
  Unity Free &
  Gratis &
  \begin{tabular}[c]{@{}l@{}}Unity es un motor de videojuego\\  multiplataforma creado por Unity Technologies.\\  Unity está disponible como plataforma de desarrollo\\  para Microsoft Windows, Mac OS, Linux. \\ La plataforma de desarrollo tiene soporte de compilación\\  con diferentes tipos de plataformas\end{tabular} \\ \hline
\cellcolor[HTML]{C0C0C0}Motor de Desarrollo &
  Unreal Engine 4 &
  Gratis &
  \begin{tabular}[c]{@{}l@{}}Unreal Engine es un motor de juego creado por la compañía\\  Epic Games, mostrado inicialmente en el shooter en primera\\  persona Unreal en 1998. \\ Aunque se desarrolló principalmente para los shooters en primera\\  persona, se ha utilizado con éxito en una variedad de otros géneros.\end{tabular} \\ \hline
\end{tabular}%
}
\caption{Plataforma de desarrollo.
}
\label{tab:t22}
\end{table}

Tomando en cuenta la infraestructura que posee la Escuela Superior de Medicina del Instituto Politécnico Nacional, específicamente el Departamento de Computación se optó por usar el sistema operativo Windows 10 Pro y Unity ®.

\section{Hardware}
De acuerdo a las especificaciones del software seleccionado se recomienda usar:\\
\begin{itemize}
\item Tarjeta Gráfica: NVIDIA GTX 1060 / AMD Radeon RX 480 o mejor
\item Tarjeta Gráfica Alternativa: NVIDIA GTX 970 / AMD Radeon R9 290 o mejot
\item CPU: Intel i5-4590 / AMD Ryzen 5 1500X o mejor
\item Memoria: 8GB+ RAM
\item Salida de Video: Compatible HDMI 1.3 video output
\item USB puertos: 3x USB 3.0 y  1x USB 2.0
\end{itemize}
Y en caso de no contar con estos elementos, el mínimo es:
\begin{itemize}
\item Tarjeta Gráfica: NVIDIA GTX 1050Ti / AMD Radeon RX 470 o mejor
\item Tarjeta Gráfica Alternativa: NVIDIA GTX 960 / AMD Radeon R9 290 o mejor
\item CPU: Intel i3-6100 / AMD Ryzen 3 1200, FX4350 o mejor
\item Memoria: 8GB+ RAM
\item Salida de Video: Compatible HDMI 1.3 video output
\item Puertos USB: 1x USB 3.0 y 2x USB 2.0
\end{itemize} 

\section{Viabilidad}
También se analizó la factibilidad del proyecto en general. Desde el punto de vista técnico se realizó una evaluación de la tecnología actual existente y la posibilidad de utilizarla en el desarrollo del sistema. Además de DirectX versión 11 el cuadro \ref{tab:t23} muestra los recursos técnicos necesarios para la ejecución correcta del software:\\
\begin{table}[H]
\centering
\begin{tabular}{|c|c|l|}
\hline
\rowcolor[HTML]{9B9B9B} 
\textbf{Cantidad} &
  \textbf{Recursos} &
  \multicolumn{1}{c|}{\cellcolor[HTML]{9B9B9B}\textbf{Características}} \\ \hline
\cellcolor[HTML]{C0C0C0}1 &
  Computadora Personal de Escritorio &
  \begin{tabular}[c]{@{}l@{}}Tarjeta gráfica discreta, RX480\\ Memoria RAM 16 Gb\\ 5 Puertos USB\\ Procesador de 4 núcleos o mayor\end{tabular} \\ \hline
\cellcolor[HTML]{C0C0C0}1 &
  Sistema de realidad Virtual Oculus Rift &
  \begin{tabular}[c]{@{}l@{}}Visor HMD, controles Touch \\ \\ Sensores Touch.\end{tabular} \\ \hline
\end{tabular}
\caption{ Recursos Técnicos.}
\label{tab:t23}
\end{table}
Económicamente, se determinaron los recursos para desarrollar el sistema así como la comparativa con el uso de cuerpos para su examinación y estudio. Después de un análisis e investigación de los costos con la dirección del Área de Morfología en la Escuela Superior de Medicina bajo la asesoría del Dr. Macias Rios se determinó que el costo que se tiene para el traslado, mantenimiento, uso e inhumación de los cuerpos es de \$40,000.00 c/u, como se ve en el cuadro \ref{tab:t24}.
\begin{center}
\begin{table}[H]
\centering
\begin{tabular}{|
>{\columncolor[HTML]{C0C0C0}}c |l|}
\hline
\multicolumn{2}{|c|}{\cellcolor[HTML]{9B9B9B}\textbf{Costo de uso de cuerpos.}} \\ \hline
Traslado, mantenimiento, uso e inhumación           & \$40,000.00 c/u           \\ \hline
Total                                               & \$40,000.00 c/u           \\ \hline
\end{tabular}
\caption{Costo de cálculo de uso de cuerpos.}
\label{tab:t24}
\end{table}
\end{center}
En el caso del desarrollo e implementación del proyecto se consideró la depreciación, como se observa en el cuadro \ref{tab:t26}.


Para ofrecer una experiencia aceptable al momento del uso del equipo de Realidad Virtual y el software se proponen los elementos del cuadro \ref{tab:t27}.


Además, el sistema de Realidad Virtual con sus componentes tiene un costo que se muestra en el cuadro \ref{tab:t28}.

Se estimaron los suelos de programador y modelado, como se observa en el cuadro \ref{tab:t29}.

Los servicios estimados se muestran en el cuadro \ref{tab:t210} y en el cuadro \ref{tab:t211} se muestra la suma total y como resultado se obtiene el costo total del proyecto, que se estima en: \$326,316.86.

En resumen, el costo de usar nueve cuerpos sería de \$360,000.00 y el del proyecto de \$326,318.00, con lo cual se puede considerar viable económicamente.

\section{Análisis de la plataforma Unity ®}
Unity ® es un motor de desarrollo de videojuegos multiplataforma desarrollado por Unity ® Technologies. Con él se pueden crear videojuegos, simuladores, software de realidad virtual y aumentada. Puede generar código para computadoras de escritorio, portátiles, consolas de videojuegos, Smart TV y otros dispositivos móviles. Ofrece una API de scripting en C\#. En la figura 8 se puede ver el entorno en general.\\
