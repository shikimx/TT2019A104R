\chapter{Conlusiones y Trabajo a Futuro}

\section{Productos Logrados}
Se construyó un prototipo de sistema de  software que proporciona una experiencia demostrativa en realidad virtual, con algunas características y 
elementos del sistema digestivo, el cual  permite la participación de un usuario utilizando una computadora de forma local. El sistema de hardware 
que proporciona el entorno virtual está conformado por los controles Oculus Touch ® , el visor de Realidad Virtual Oculus Rift ®.

\subsection{Complicaciones del desarrollo sin metodología}
El proyecto comenzó su desarrollo sin tomar en cuenta la metodología de desarrollo OpenUP, lo cual causó que el trabajo se viera mermado en cuanto a su documentación y desarrollo.\\
El desarrollo de un sistema mediante OpenUP puede ser una herramienta poderosa para el desarrollo de software ya que dentro de ella se desarrollan micro incrementos en el desarrollo del software y además cumple con los principios del Manifiesto Ágil\cite{beck2001manifesto}.\\
La problemática surge cuando se intenta implementar en un proyecto de software diferente al cual usualmente se desarrolla, por ejemplo un Sistema de Calendarización de Trabajos terminales o un Generador de Páginas Web, difiere en cuanto a su desarrollo ya que se usan herramientas y elementos diferentes, tales como el desarrollo en el motor Unity ®, el diseño de los modelos 3D y el uso de elementos propietarios de Oculus ® dificultan la tarea de documentación y desarrollo bajo lineamientos por más ligeros que sean debido a la naturaleza del proyecto.\\
Es por eso que se buscó como alternativa una metodología que tome en cuenta las características inherentes de un software que su núcleo está arraigado a la realidad virtual haciéndolo un software multimedia.\\
Debido al cambio de enfoque de desarrollo de la metodología las tareas de reingeniería son suplementadas por más tiempo de desarrollo y refinamiento de los elementos anteriores a estos dentro del cronograma de actividades.\\

\section{Trabajo a futuro}
Este proyecto puede continuar si se realiza el desarrollo y la implementaci\'on de los d\'emas sistemas del cuerpo humano, as\'i ampliando su alcance para su uso en la Escuela Superior de Medicina,
as\'i mismo se mejorar la interacci\'on con los modelos en 3D y la manera de exhibici\'on de las caracter\'isticas de los \'organos del cuerpo humano.
Tambi\'en se realizar\'ia la evaluaci\'on de los modelos por personal calificado para as\'i relizar la reedici\'on y perfeccionamiento de los modelos en 3D.
Finalmente el empaquetado del software y su districi\'on en las plataformas necesarias para tener el alcance necesario, un ejemplo de estas puede ser la misma tienda de 
Oculus.

\section{Conclusión}
Como sucede con la mayoría de los sistemas de software, quizás al seguir siendo un campo de desarrollo tan reciente, sea esa la razón de porqué 
la gente suele infravalorar la cantidad de tiempo, recursos y esfuerzo que supone la puesta en obra. A diferencia de otros trabajos terminales, 
este ha sido una incursión fuera del software “tradicional” que se propone frecuentemente en la Escuela Superior de Cómputo, en este caso, 
el tiempo invertido en el desarrollo, investigación de un área completamente ajena a mi formación y tiempo invertido en la experiencia de usuario, 
áreas en las cuales puedo decir que es poco común que se centren las unidades académicas de la escuela, aunado a esto dadas las actividades que se 
requieren para un proyecto de esta naturaleza, el tiempo y los recursos así como el esfuerzo necesario implicaría quizás como en muchos desarrollos 
de este tipo un equipo de desarrollo completo para llegar a el estado “final” del mismo.\\ 
Podría concluir que la necesidad de crear un sistema innovador y poco recurrido para los trabajos terminales es importante a la hora de desarrollar cualquier cosa.\\
En general las habilidades que no forman parte explícitamente el desarrollo de software no suelen ser tomados en cuenta y frecuentemente son tratadas 
como no necesarias para el estudiante y profesores. Quiero mencionar que, sobre todo, y como en los sistemas de software tradicionales, la planeación 
y diseño es fundamental, y que no se debe de dejar de lado, incurrir en un sistema con tecnología tan nueva y utilizada en un menor grado para Trabajos 
Terminales complica el desarrollo del mismo.\\
La abstracción de un sistema de realidad virtual, el cual tiene una línea muy delgada para ser catalogado un videojuego y muchas veces es catalogado 
como un serious game, el cual al momento de su desarrollo se pueden definir elementos concisos y enlistarlos para su planeación, pero cada feature debe 
nacer de un concepto previo, debe tener un propósito y consideración. Es importante tener enfoque en el crecimiento del proyecto. En el desarrollo de 
experiencias de realidad virtual, la composición de elementos es una forma natural para la construcción de dicho sistema, esto facilita un crecimiento 
del proyecto, pero es un arma de doble filo ya que al permitir esta libertad da paso a vicios y concentraciones, de ahí que todo venga de una planeación 
dividiendo los elementos de software y multimedias y se pensado a la posible expansión.\\
El proyecto, si bien es pequeño comparado a la mayoría de producciones comerciales que se encuentran en el mercado, hace ver claramente que el camino 
a seguir para el desarrollo de sistemas basados en realidad virtual son un campo que definitivamente requiere de una amplia gama de personas hábiles tanto 
en diseño, tecnología, investigación y planeación. Es un campo que requiere muchísimo desarrollo en cuanto a sus metodologías y formas de documentación para 
sistemas no “tradicionales”, aunque tengo la certeza de mientras más avance la tecnología y se creen más sistemas de esta índole el desarrollo irá creciendo, 
haciéndose cada vez más profesional y disponible para todos.\\
