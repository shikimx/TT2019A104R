\chapter{Marco Conceptual}

\section{Realidad virtual}
Algunos autores definen así la Realidad Virtual.\\
\newline
\textit{“La realidad virtual (RV) es una simulación tridimensional generada o asistida comúnmente por computadora de algún aspecto del mundo real o ficticio, en el cual 
el usuario tiene la sensación de pertenecer a ese ambiente sintético o interactuar con él”}\cite{web6}\\ 
\textbf{Corrado Padila Érica}\\
\newline
“Realidad Virtual: gráficos 3D en entornos inmersivos que usan I/O
artefactos como guantes, cascos, etc. en busca de mayores grados de iteración
con el ambiente virtual”\cite{web7}\\ 
\textbf{Lozano Miguel, Calderón}\\
\newline
"Realidad Virtual es una forma en que los seres humanos puedan
visualizar, manipular e interactuar con las computadoras y datos extremadamente
complejos".\cite{web8}\\
\textbf{Isdale, Jerry}\\
\newline
“Un sistema interactivo capaz de crear una simulación que implique a varios de los sentidos del ser humano, generados por una computadora, explorable, visualizable y manipulable 
en tiempo real; este bajo la forma de imágenes y sonidos, estos, dando la sensación de presencia en el entorno generado”\cite{web9}\\
\textbf{Levis, Diego}\\
\newline
Esta última ha sido la definición que se ha tomado para el desarrollo del proyecto del trabajo terminal, asimismo se puede concluir que todos los autores coinciden en que la 
realidad virtual es un mundo simulado en el que el usuario puede interactuar en tiempo real por medio
de dispositivos o computadoras que logran un efecto artificial e inmersivo en el que se pueden manipular objetos.

\section{Relaidad Virtual como Apoyo a la Enseñanza-Aprendizaje}
%Algo breve de lo que ya se tiene


\section{Obtención de temas}
En la siguiente sección se desarrollan los temas obtenidos necesarios para los componentes multimedia del sistema digestivo del cuerpo humano.\\

\section{El Sistema Digestivo}
El cuerpo humano es una estructura compleja y altamente organizada, formada por células que trabajan juntas para realizar funciones específicas necesarias 
para mantener la vida.[ 33 ] La biología del cuerpo humano incluye:
\begin{itemize}
	\item Fisiología (cómo funciona el cuerpo)
	\item Anatomía (cómo se estructura el cuerpo)	
\end{itemize}
La anatomía está organizada por niveles, desde los componentes más pequeños de las células hasta los órganos más grandes, así como su relación con otros órganos.\\
La anatomía general estudia los órganos tal como aparecen a simple vista o en una disección del cuerpo.\\
La anatomía celular es el estudio de las células y sus componentes, los cuales pueden observarse solo con la ayuda de técnicas e instrumentos especiales como los microscopios.\\
La anatomía molecular (a menudo llamada biología molecular) estudia los componentes más pequeños de las células al nivel bioquímico.\\
\begin{figure}[H]
	\begin{center}
 		\includegraphics[width = .7\textwidth]{source/images/image22.png}
 		\captionof{figure}{\label{fig:im23}Gráfico ejemplificando los sistemas del cuerpo humano.\cite{web11}}
	\end{center} 
\end{figure}
La cavidad abdominal es el espacio corporal que ocupa la región del abdomen, ubicada entre el diafragma y la abertura de la pelvis. 
Es la cavidad más grande del cuerpo humano y contiene los principales órganos del aparato digestivo, urinario y genital.
Para su estudio y evaluación clínica en el campo de la medicina, el abdomen debe ser dividido topográficamente de forma externa en 9 
cuadrantes o regiones, utilizando cuatro líneas trazadas imaginariamente, dos verticales y dos horizontales.\cite{web12}
\begin{figure}[H]
	\begin{center}
 		\includegraphics[width = .3\textwidth]{source/images/image56.png}
 		\captionof{figure}{\label{fig:im24}Diagrama que muestra las diferentes cavidades del cuerpo humano.}
	\end{center} 
\end{figure}
Ahora se presenta un resumen conceptual de los elementos que conforman a la cavidad abdominal, esto para tener claro cuales son los elementos 
que la componen el mismo para así tener en claro cómo se encuentra conformado y poseer un mayor entendimiento de los conceptos básicos.\\
Ahora se presenta un resumen conceptual de los elementos que conforman a la cavidad abdominal, esto para tener claro cuales son los elementos 
que la componen el mismo para así tener en claro cómo se encuentra conformado y poseer un mayor entendimiento de los conceptos básicos.\\
\begin{itemize}
	\item El abdomen es la región que se encuentra entre el tórax y la pelvis. El diafragma separa las estructuras del tórax de las del abdomen. 
	\item La abertura superior de la pelvis comunica la pelvis con la cavidad abdominal.
	\item La pared abdominal se encuentra compuesta por piel, tejido subcutáneo, planos musculares con sus aponeurosis y fascias y peritoneo parietal.
	\item La cavidad abdominal se encuentra integrada por la cavidad peritoneal, el retroperitoneo y las vísceras peritonizadas.
	\item La cavidad peritoneal se encuentra limitada por las láminas visceral y parietal del peritoneo.
	\item El retroperitoneo está integrado por todas las estructuras anatómicas que se encuentran por detrás de la lámina parietal posterior del peritoneo.
	\item La cavidad abdominal incluye estructuras pertenecientes a los sistemas digestivo, endocrino, vascular, nervioso y urinario.
	\item Las vísceras sólidas de la cavidad abdominal son el hígado, el bazo, el páncreas, los riñones y las glándulas suprarrenales.
	\item Las huecas son el tubo digestivo y las vías de excreción urinaria. 
	\item El tubo digestivo abdominal está integrado por el esófago (porción abdominal), el estómago, el duodeno, el yeyuno, el íleon y las porciones del colon (ciego, apéndice vermiforme, colon ascendente, transverso, descendente y sigmoide).
	\item El sistema vascular incluye las arterias ramas de la porción abdominal de la aorta, las venas tributarias de la cava inferior y la vena porta hepática y los linfáticos.
	\item La porción abdominal de la arteria aorta se extiende desde el hiato aórtico del diafragma y sus ramas terminales son las arterias ilíacas comunes. Se encuentra en situación prevertebral desplazada hacia la izquierda de la columna lumbar. 
	\item La vena cava inferior se forma por la anastomosis de ambas ilíacas comunes y asciende en situación prevertebral desplazada a la derecha hasta el orificio de la vena cava en el diafragma.
	\item La vena porta hepática se forma detrás de la cabeza del páncreas por la anastomosis de las venas esplénica (que ya recibe como afluente a la vena mesentérica inferior) y mesentérica superior, y se dirige al hígado dentro del omento menor.
	\item Los troncos linfáticos de los miembros inferiores, la pelvis y el abdomen confluyen detrás de la cabeza del páncreas donde se origina el conducto torácico.Este asciende y atraviesa el diafragma por el hiato aórtico en dirección al tórax
\end{itemize}
Aunado a esto se incorporan todos los elementos del sistema digestivo, expuestos por el/los autores de los diferentes materiales en los cuales, pero no limitados 
a estos, se realizó la investigación del sistema, cabe mencionar que en algunas figuras que se muestran se incluyen elementos que no son parte de los componentes 
del sistema digestivo pero estos se incluyen en las figuras debido a que son parte de la cavidad abdominal, o en su defecto el órgano en el cual se está centrando 
el desarrollo del modelo se encuentra demasiado cerca de un órgano u órganos contiguos para ser incluido en la figura individualmente.\\
A continuación se muestra una representación del sistema digestivo, está contienen la información de cuáles son los órganos y elementos que lo conforman, estos serán 
varios ya que se han tomado como guía, estos elegidos mediante a una entrevista de estudiantes de medicina como material que se ha utilizado para el estudio del cuerpo 
humano, para la realización de modelos en 3D, asimismo, ejemplifica el material que dispone pero no limitado a el alumnado para el estudio del sistema en cuestión.\\
Aunado a esto se incorporan la mayoría de los elementos del sistema digestivo los cuales se encuentran en la cavidad abdominal estos expuestos por autores de los 
diferentes materiales en los cuales, pero no limitados a estos, se realizó la investigación del sistema.\\
\begin{figure}[H]
	\begin{center}
 		\includegraphics[width = .3\textwidth]{source/images/image72.png}
 		\captionof{figure}{\label{fig:im25}Órganos digestivos in situ (El epiplón mayor ha sido parcialmente eliminado o reflejado)\cite{rohen2018anatomy}}
	\end{center} 
\end{figure}

\section{Modelado 3D}
En general, independientemente de la disciplina, el proceso de modelado es una simplificación de un objeto para su posterior estudio o representación. Así, podemos hablar de modelos matemáticos que simplifican fenómenos físicos, o modelos meteorológicos para la predicción del tiempo atmosférico, etc. Un modelo geométrico define la información sobre la forma (geometría) de un determinado objeto. Las simplificaciones que se realicen en su definición vendrán determinadas por diferentes factores como el método de representación utilizado, operadores empleados o nivel de detalle.\cite{web13} \\

Se puede definir el proceso de modelado geométrico tridimensional como el encargado de crear modelos consistentes que puedan ser manejados algorítmicamente en un computador. Este proceso de construcción se aborda en diferentes etapas, partiendo típicamente de entidades básicas y aplicando una serie de operadores sobre ellas. Estas entidades básicas pueden ser primitivas geométricas (calculadas de forma algorítmica o mediante una ecuación matemática) u obtenidas mediante un dispositivo de captura (escáner 3D).\\

Existen multitud de técnicas de modelado 3D. En una primera taxonomía de alto nivel podemos hacer una categorización dependiendo de si el modelado se centra en definir únicamente las características del contorno del objeto, los siguiente son los mas usados:\\
\begin{itemize}
\item \textbf{Modelado Sólido:} también conocidos como de Geometría Sólida Constructiva (CSG Constructed Solid Geometry). Los modelos sólidos definen el volumen del objeto que representan, y en muchos casos indican incluso el centro de masas, la densidad del material interna, etc. Se utilizan en fabricación por computador y en aplicaciones médicas e industriales.
\item \textbf{Modelado de Contorno:} también conocidos como de Representación de Contorno (B-Rep - Boundary Representation). Los modelos de contorno únicamente representan la superficie límite del objeto (de forma conceptual, la "cáscara"). Son más fáciles de definir y modificar. Además, lo interesante para la representación del objeto es su apariencia exterior (en los casos donde interesa el interior simplemente se aproxima, como en el caso del SubSurfaceScattering). Prácticamente todos los paquetes de diseño y animación (incluido Blender) empleados en síntesis de imagen y en aplicaciones interactivas emplean este tipo de modelos.
\end{itemize}
Para  cubrir las necesidades de los modelos 3D de los órganos del sistema digestivo se ha opto por que estos fueran realizados en el modelado de contorno por su facilidad de desarrollo y ligereza de carga en el renderizado en el momento de la implementación de estos en el sistema de realidad virtual.\\

\section{Generacion de Entorno 3D}
El entorno 3D es en donde el usuario se encontrará al ingresar al sistema de realidad virtual, para, este se ha realizado para dar la sensación de encontrarse en un ambiente médico.\\

Se utilizaron modelos ya realizados por un autor adquiriendo los derechos de uso ya que la realización de estos no se consideran parte integral del desarrollo del Trabajo Terminal, escrito esto no se quiere demeritar la necesidad de hacer el usuario ya que, como se ha mencionado en secciones anteriores, se tiene énfasis en la experiencia del usuario para que la inmersión del usuario sea la mayor posible.\\

A continuación se muestran capturas del entorno 3D como fue implementado dentro el motor de desarrollo Unity.\\

\begin{figure}[H]
	\begin{center}
 		\includegraphics[width = .5\textwidth]{source/images/image63.png}
 		\captionof{figure}{\label{fig:im31}Entorno 3D vista normal}
	\end{center} 
\end{figure}

\begin{figure}[H]
	\begin{center}
 		\includegraphics[width = .5\textwidth]{source/images/image53.png}
 		\captionof{figure}{\label{fig:im32} Entorno 3D vista externa de la escena}
	\end{center} 
\end{figure}

\begin{figure}[H]
	\begin{center}
 		\includegraphics[width = .5\textwidth]{source/images/image16.png}
 		\captionof{figure}{\label{fig:im33}Entorno 3D  vista principal}
	\end{center} 
\end{figure}

\section{Modelos 3D de los Organos}
Los componentes multimedia a desarrollar en modelos 3D los cuales son miembros del sistema digestivo del ser humano, 
el sistema digestivo incluye a los órganos del tubo alimenticio y glándulas de secreción exocrina y endocrina.\\

\subsection{Glándulas Salivales}
 continuación se muestran las figuras del resultado final del desarrollo de las glándulas salivales del sistema digestivo 
 en el software de modelado en 3D llamado “Blender”, este fue realizado basado en el material anteriormente provisto.\\
\begin{figure}[H]
	\begin{center}
 		\includegraphics[width = .5\textwidth]{source/images/image41.png}
 		\captionof{figure}{\label{fig:im34}Modelo 3D de las glándulas salivales}
	\end{center} 
\end{figure}

\subsection{Cavidad oral y faringe}
A continuación se muestran las figuras del resultado final del desarrollo de la cavidad oral del sistema digestivo en el software de modelado en 3D llamado “Blender”, este fue realizado basado en el material anteriormente provisto.\\
\begin{figure}[H]
	\begin{center}
 		\includegraphics[width = .5\textwidth]{source/images/image14.png}
 		\captionof{figure}{\label{fig:im35}Modelo 3D de la cavidad oral}
	\end{center} 
\end{figure}

\subsection{Esófago}
A continuación se muestran las figuras del resultado final del desarrollo del esófago del sistema digestivo en el software de modelado en 3D llamado “Blender”, este fue realizado basado en el material anteriormente provisto.\\
\begin{figure}[H]
	\begin{center}
 		\includegraphics[width = .5\textwidth]{source/images/image25.png}
 		\captionof{figure}{\label{fig:im36}Modelo 3D del esófago}
	\end{center} 
\end{figure}

\subsection{Estómago}
A continuación se muestran las figuras del resultado final del desarrollo del estómago del sistema digestivo en el software de modelado en 3D llamado “Blender”, este fue realizado basado en el material anteriormente provisto.\\
\begin{figure}[H]
	\begin{center}
 		\includegraphics[width = .5\textwidth]{source/images/image42.png}
 		\captionof{figure}{\label{fig:im37} Modelo 3D del estómago }
	\end{center} 
\end{figure}

\subsection{Intestino delgado}
A continuación se muestran las figuras del resultado final del desarrollo del intestino delgado del sistema digestivo en el software de modelado en 3D llamado “Blender”, este fue realizado basado en el material anteriormente provisto.\\
\begin{figure}[H]
	\begin{center}
 		\includegraphics[width = .5\textwidth]{source/images/image69.png}
 		\captionof{figure}{\label{fig:im38}Modelo 3D del intestino delgado}
	\end{center} 
\end{figure}

\subsection{Hígado}
A continuación se muestran las figuras del resultado final del desarrollo del hígado del sistema digestivo en el software de modelado en 3D llamado “Blender”, este fue realizado basado en el material anteriormente provisto.\\
\begin{figure}[H]
	\begin{center}
 		\includegraphics[width = .5\textwidth]{source/images/image17.png}
 		\captionof{figure}{\label{fig:im39}Modelo 3D del hígado}
	\end{center} 
\end{figure}

\subsection{Páncreas}
A continuación se muestran las figuras del resultado final del desarrollo del páncreas del sistema digestivo en el software de modelado en 3D llamado “Blender”, este fue realizado basado en el material anteriormente provisto.\\
\begin{figure}[H]
	\begin{center}
 		\includegraphics[width = .5\textwidth]{source/images/image19.png}
 		\captionof{figure}{\label{fig:im310}Modelo 3D del páncreas}
	\end{center} 
\end{figure}

\subsection{Vesícula Biliar}
A continuación se muestran las figuras del resultado final del desarrollo de la vesícula biliar del sistema digestivo en el software de modelado en 3D llamado “Blender”, este fue realizado basado en el material anteriormente provisto.\\
\begin{figure}[H]
	\begin{center}
 		\includegraphics[width = .5\textwidth]{source/images/image26.png}
 		\captionof{figure}{\label{fig:im312}Modelo 3D de la vesícula biliar}
	\end{center} 
\end{figure}

\subsection{Intestino Grueso y Ano}
A continuación se muestran las figuras del resultado final del desarrollo del intestino grueso y ano del sistema digestivo en el software de modelado en 3D llamado “Blender”, este fue realizado basado en el material anteriormente provisto.\\
\begin{figure}[H]
	\begin{center}
 		\includegraphics[width = .5\textwidth]{source/images/image20.png}
 		\captionof{figure}{\label{fig:im313}Modelo 3D del intestino grueso}
	\end{center} 
\end{figure}

\section{Modelo del sistema digestivo unificado}
A continuación se muestran las figuras del resultado final del desarrollo del sistema digestivo en el software de modelado en 3D llamado “Blender”, este fue realizado reuniendo todos los modelos de órganos y elementos individuales creados con anterioridad.\\
\begin{figure}[H]
	\begin{center}
 		\includegraphics[width = 1\textwidth]{source/images/image24.png}
 		\captionof{figure}{\label{fig:im314}Modelo 3D del Sistema digestivo}
	\end{center} 
\end{figure}

\section{Evaluación de modelos 3D por personal calificado}
Debido al tiempo de desarrollo, el cual tomó más de lo planteado, de los modelos antes expuestos en la sección anterior no fue posible concretar una cita 
para su evaluación con el personal calificado  de la Escuela Superior de Medicina en las fechas previamente planteadas.\\
Se esperaba poder tener una reunión en fechas posteriores pero la situación epidémica que se ha desarrollado en el país y 
limitaciones impuestas por las autoridades hicieron imposible la evaluación de los modelos desarrollados.\\
Esto no significa que no se haya hecho bajo rigor alguno, sólo se utilizaron materiales de medicina impresos, así como 
referencias en video de disecciones del sistema digestivo, esto para estar lo más familiarizado posible, como estudiante 
de ingeniería en sistemas computacionales, al momento de desarrollar dichos modelos.\\
% (Aquí puede decir lo de unity y demás cosas que se requieren para hacer lo que usted ya hizo)
%(Tratemos de usar lo que se tiene y solo añadir si es indispensable, recortar si es necesario para añadir a Apendices, como lo de las imagenes del aparato digestivo que estaban en el primer documento)