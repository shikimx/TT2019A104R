\begin{center}
\chapter{Referencias y Glosario}
\end{center}
\newpage

\section{Glosario}
\begin{itemize}
 \item \textbf{VR:} Virtual Reality, en español, Realidad Virtual
 \item \textbf{RV:} Realidad Virtual
 \item \textbf{ICs:} Tecnologías de la Información y la Comunicación
 \item \textbf{WIMP:} del inglés window-icon-menu-pointing device
 \item \textbf{MRI:} Imagen por resonancia magnética del inglés Magnetic Resonance Imaging
 \item \textbf{CT:} Computed Tomography del inglés Tomografía axial computarizada
 \item \textbf{PET:} Tomografía por emisión de positrones del inglés Positron Emission Tomography
 \item \textbf{DICOM:} Imagen digital y comunicación sobre medicina. Digital Imaging and Communication On Medicine
 \item \textbf{HMD:} Casco de realidad virtual del inglés Head Mounted Display
 \item \textbf{UML:} Lenguaje de modelado unificado del inglés Unified Modeling Language
 \item \textbf{3D:} Tres Dimensiones del inglés 3 Dimensions
 \item \textbf{I/O:} Entrada y salida del inglés Input/Output
 \item \textbf{UX:} UX (por sus siglas en inglés User eXperience) o en español Experiencia de Usuario, es aquello que una persona percibe al interactuar con un producto o servicio. Logramos una buena UX al enfocarnos en diseñar productos útiles, usables y deseables, lo cual influye en que el usuario se sienta satisfecho, feliz y encantado.
 \item \textbf{VOD:} Del inglés  Video On Demand, El vídeo bajo demanda, o televisión a la carta, es un servicio OTT de televisión. Esta modalidad de difusión de contenidos multimedia, permite al usuario acceder a un contenido concreto, en el momento que lo solicita, visualizandolo en línea en su dispositivo.
 \item \textbf{SDK:} Un kit de desarrollo de software (en inglés, software development kit o SDK) es generalmente un conjunto de herramientas de desarrollo de software que permite a un desarrollador de software crear una aplicación informática para un sistema concreto, por ejemplo ciertos paquetes de software, entornos de trabajo
 \item \textbf{RUP:} El Proceso Unificado de Rational o RUP (por sus siglas en inglés de Rational Unified Process) es un proceso de desarrollo de software desarrollado por la empresa Rational Software, actualmente propiedad de IBM.
\end{itemize}
