\chapter{Introducción}

\section{Introducción}
Este documento es el reporte técnico final del trabajo terminal titulado “Sistema de realidad virtual del cuerpo humano para el estudio del sistema digestivo” con número de registro TT: 2019-A104.

En el presente capítulo se habla del problema identificado, por qué se considera como tal y cómo es que se ayudó a resolver el problema planteado mediante la ingeniería en sistemas computacionales. También se menciona que se obtiene al concluir con este trabajo terminal, tales como el prototipo del sistema.\\
\newline
En el capítulo \textbf{II Análisis} se mostrarán todos los diagramas y documentos generados al analizar y generar un diseño del sistema que se estará desarrollando. Aquí se encuentra la arquitectura general del sistema, y los modelos gráficos de apoyo presentados en el Análisis Estructurado Moderno.\\
\newline
En el capítulo \textbf{III Diseño} se describe el trabajo generado en el desarrollo del documento hasta el mes de mayo de 2020 para TT2.\\
\newline
En el capítulo \textbf{IV Verificación y Pruebas} se muestran pruebas hechas sobre las implementaciones del sistema siguiendo un guión para la prueba.
 \newline\\
En el capítulo \textbf{V Conclusión} se muestran los resultados obtenidos y experiencias para mejorar el proceso, así como la vertiente para continuar con el trabajo y las conclusiones del integrante.
\newline\\
Finalmente, se encuentran las referencias de todos los recursos empleados para dar soporte y estructura a este Trabajo Terminal, y en los apéndices se anexan elementos extra que dan información más detallada sobre lo que aquí fue realizado.\\
\vfill 

\newpage

\section{Detecci\'oon del problema} 

Durante 2019 se entrevistó al Dr. Rios Macias jefe del área de morfología de la Escuela Superior de Medicina del Instituto Politécnico Nacional y comentó que “los medios que se utilizan para el estudio del cuerpo humano principalmente son medios impresos tradicionales, así como el uso de cuerpos para su disección y análisis posterior”. El uso de cuerpos para su disección tiene un alto costo que incluye el mantenimiento del cuerpo en las instalaciones, el mantenimiento de las instalaciones, y la inhumación de los cuerpos.
\\
\newline
\section{Propuesta de Solución}

Se elaboró un sistema de realidad virtual del sistema digestivo del cuerpo humano que permite interactuar con modelos tridimensionales. La intención es sentar las bases para un sistema de apoyo al aprendizaje que sea más práctico \cite{moore1995learning}, sin sustituir a ningún método de estudio tradicional.
\\
\newline
\section{Justificación}

El sistema tiene los siguientes beneficios para docentes y alumnos: 
\begin{itemize}
\item Fuentes confiables. Se usaron textos médicos y sitios web especializados
\item Uso de realidad virtual\cite{norton1994integrating}.
\end{itemize}

El sistema es un ejemplo de cómo se pueden actualizar las herramientas educativas, como las que se pueden encontrar en Statista \cite{web1}, y que representan un mercado de aproximadamente 18.8 mil millones de dólares para el año 2020. Además, Nielsen\cite{} muestra los datos sobre la expectativa de adopción de la realidad virtual y aumentada en diferentes continentes, como se muestra el Figura 1 y 2.

\begin{figure}[H]
	\begin{center}
 		%%\includegraphics[width = 0.5\textwidth]{images/ipn.png}
 		\captionof{figure}{\label{fig:IPN}Descripción de la imagen} 
	\end{center} 
\end{figure}

\begin{figure}[H]
	\begin{center}
 		%%\includegraphics[width = 0.5\textwidth]{images/ipn.png}
 		\captionof{figure}{\label{fig:IPN}Descripción de la imagen} 
	\end{center} 
\end{figure}

\subsection{Generación del concepto.}
