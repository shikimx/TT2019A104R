\begin{center}
\chapter{Verificación y Pruebas}
\end{center}
\newpage

\section{Mecanismos de medición para la fase de pruebas}
Para este trabajo se realizaron dos tipos de prueba: pruebas de experiencia de usuario mediante las cuales, se busca obtener retroalimentación por parte de potenciales usuarios, de manera que el desarrollador pueda identificar errores y/o posibles cambios que pudieran hacerse al producto. \\

Como mecanismo de medición para las pruebas de experiencia de usuario, se emplea el método de validación de la metodología de Design Sprint de Google ® Ventures\cite{web16}. El proceso de validación se compone de dos partes:\\
\begin{enumerate}
    \item Pruebas de usuario.
    \item Retroalimentación de los stakeholders.
\end{enumerate}
Después de la finalización del prototipo se procede a realizar pruebas. Una simple prueba de usuario permite descubrir información valiosa rápidamente. Ayuda a responder preguntas como ¿Qué es lo que los usuarios disfrutan o no gustan del prototipo? ¿Qué les gustaría que mejorara? \\

Después se continúa con la validación de los interesados o accionistas (stakeholders). Estas personas son el director de la Escuela Superior de Medicina, profesores del área de morfología y los alumnos de la misma. \\

Para motivos de este trabajo y los tiempos que se viven, el stakeholder será el mismo estudiante que presenta este trabajo, ya que resulta imposible hacer pruebas con los usuarios potenciales debido a la situación de epidemia nacional por el virus SARS-COV-2.\\

De igual forma se tomará en cuenta la retroalimentación de los sinodales y directores del proyecto mismo. Así, la revisión y aprobación de estos es esencial para que la validación se considere exitosa.\\

\section{Pruebas de Entorno 3D con headset de R.V.}
A continuación, se mostrarán las pruebas no automatizadas realizadas a las características, cada prueba está relacionada a un feature o features específicos, los cuales fueron probados conforme fueron desarrollados bajo el flujo de trabajo Git Flow\cite{pathania2017elements}.\\ 

Se detalla que es lo que se espera conseguir y bajo qué condiciones se consigue este objetivo, así como se indica si la característica ha pasado la prueba o la ha fallado. Así mismo se incluyen capturas de pantallas que muestran el funcionamiento de la característica y observaciones particulares a la prueba, indicando según lo requiera detalles de implementación o configuración.\\

\subsection{Configuración de entradas}


